\documentclass[a4paper,10pt]{article}
%\documentclass[a4paper,10pt]{scrartcl}

\usepackage{xltxtra}
\usepackage{../mystyle}

\setromanfont[Mapping=tex-text]{Linux Libertine O}
% \setsansfont[Mapping=tex-text]{DejaVu Sans}
% \setmonofont[Mapping=tex-text]{DejaVu Sans Mono}

\title{\sc Algebra I \\ \Large Blatt 5}
\author{Thorben Kastenholz \\ Jendrik Stelzner}
\date{\today}

\begin{document}
\maketitle





\section{}


\subsection{}
Für $f,g : V \to k$ ist für alle $v \in V$
\[
 (fg)(v) = 0 \Leftrightarrow f(v)g(v) = 0 \Leftrightarrow f(v) = 0 \text{ oder } g(v) = 0.
\]
Insbesondere ist daher für alle $f,g \in \mc{P}(V)$
\begin{align*}
 \mc{V}(fg)
 &= \{v \in V \mid (fg)(v) = 0\} \\
 &= \{v \in V \mid f(v) = 0 \text{ oder } g(v) = 0\} \\
 &= \{v \in V \mid f(v) = 0\} \cup \{v \in V \mid g(v) = 0\} \\
 &= \mc{V}(f) \cup \mc{V}(g).
\end{align*}


\subsection{}
Es ist
\begin{align*}
 F
 &= \left(X^2 - 9\right)^2 + \left(Y^2 - 16\right)^2 + 2\left(X^2 + 9\right)\left(Y^2 - 16\right) \\
 &= \left( \left(X^2 + 9\right) + \left(Y^2 - 16\right) \right)^2 - 36X^2 \\
 &= \left(X^2 + 9 + Y^2 - 16 + 6X\right)(X^2 + 9 + Y^2 - 16 - 6X) \\
 &= \left( (X+3)^2 + Y^2 - 16 \right) \left( (X-3)^2 + Y^2 - 16 \right).
\end{align*}
Es ist daher nach dem vorherigen Aufgabenteil
\[
 \mc{V}(F) = \mc{V}\left((X+3)^2 + Y^2 - 16\right) \cup \mc{V}\left((X-3)^2 + Y^2 - 16\right)
\]
die Vereinigung zweier Kreise, siehe Abbildung \ref{fig: Vereinigung zweier Kreise}.
\begin{figure}\centering
 \begin{tikzpicture}[scale = 0.6]
  \draw[->, very thick, black] (-8.5,0) -- (8.5,0);
  \draw[->, very thick, black] (0,-5.5) -- (0,5.5);
  \foreach \x in {-8, -6, -4, -2, 2, 4, 6, 8}
   \draw (\x, 0.2) -- (\x, -0.2) node[below] {$\x$};
  \foreach \y in {-4, -2, 2, 4}
   \draw (0.2, \y) -- (-0.2,\y) node[left] {$\y$};
  \draw[thick] (-3,0) circle (4);
  \draw[thick] (3,0) circle (4);
 \end{tikzpicture}
 \caption{$\mc{V}(F)$ aus Aufgabe 1 (b).}
 \label{fig: Vereinigung zweier Kreise}
\end{figure}


\subsection{}
Wir bemerken, dass
\[
 X^2 \left( 1 - X^2 \right) - Y^2
 = -X^4 + X^2 - Y^2
 = -\left( \left(X^2 - \frac{1}{2} \right)^2 + Y^2 - \frac{1}{4} \right).
\]
Definieren wir
\[
 f : \R^2 \to \R^2, (x,y) \mapsto \left(x^2,y\right),
\]
und den Kreis
\[
 K := \left\{ (x,y) \in \R^2 \,\middle|\, \left(x - \frac{1}{2} \right)^2 + y^2 - \frac{1}{4} = 0 \right\}
\]
so ist daher
\begin{align*}
 \mc{V}\left( X^2 \left( 1 - X^2 \right) - Y^2 \right)
 &= \left\{ (x,y) \in \R^2 \,\middle|\, \left(x^2 - \frac{1}{2} \right)^2 + y^2 - \frac{1}{4} = 0 \right\} \\
 &= f^{-1}(K).
\end{align*}
Es ergibt sich daher das Bild wie in Abbildung \ref{fig: lemis carte}
\begin{figure}\centering
 \begin{tikzpicture}[scale=1.5, domain=-1:1]
  \draw[->, very thick, black] (-1.5,0) -- (1.5,0);
  \draw[->, very thick, black] (0,-1.5) -- (0,1.5);
  \foreach \x in {-1,1}
   \draw (\x, 0.1) -- (\x, -0.1) node[below] {$\x$};
  \foreach \y in {-1,1}
   \draw (0.1, \y) -- (-0.1,\y) node[left] {$\y$};
  \draw[thick,smooth,samples=1000] plot[id=oben] function {sqrt(x**2 - x**4)};
  \draw[thick,smooth,samples=1000] plot[id=unten] function {-sqrt(x**2 - x**4)};
 \end{tikzpicture}
 \caption{$\mc{V}\left(X^2 \left( 1 - X^2 \right) - Y^2\right)$ aus Aufgabe 1 (c).}
 \label{fig: lemis carte}
\end{figure}





\section{}


\subsection{}
Es ist bekannt, dass $\varphi*$ ein $k$-Algebrahomorphismus ist. Daher ist $\varphi^*$ genau dann injektiv, wenn $\ker \varphi^* = 0$. Da für alle $f \in \mc{P}(V)$
\begin{align*}
 f \in \ker \varphi^*
 &\Leftrightarrow \varphi^*(f) = 0 \\
 &\Leftrightarrow f \circ \varphi = 0 \\
 &\Leftrightarrow (f \circ \varphi)(w) = 0 \text{ für alle } w \in W \\
 &\Leftrightarrow f(y) = 0 \text{ für alle } y \in \varphi(W)
\end{align*}
ist $\ker \varphi^* = 0$ genau dann, wenn
\[
 f_{|\varphi(W)} = 0 \Leftrightarrow f = 0.
\]
Diese Äquivalenz gilt genau dann, wenn $\varphi(W)$ Zariski-dicht in $V$ liegt.


\subsection{}
Wir bemerken zunächst, dass es für alle $w, w' \in W$ eine polynomielle Funktion $f \in \mc{P}(W)$ gibt mit $f(w) \neq f(w')$. Denn wählen wir eine Basis $w_1, \ldots, w_n$ von $W$ (möglich, da $W$ endlichdimensional ist), so können wir $w$ und $w'$ eindeutig als
\[
 w = \sum_{i=1}^n \lambda_i w_i \quad \text{und} \quad w' = \sum_{i=1}^n \lambda'_i w_i
\]
schreiben. Da $w \neq w'$ gibt es $1 \leq j \leq n$ mit $\lambda_j \neq \lambda'_j$. Für
\[
 \pi : W \to k, \sum_{i=1}^n \mu_i w_i \mapsto \mu_j
\]
ist offenbar $\pi \in \mc{P}(W)$, und da $\lambda_j \neq \lambda'_j$ ist $\pi(w) \neq \pi(w')$.

Ist $\varphi$ nicht injektiv, so gibt es $w, w' \in W$ mit $\varphi(w) = \varphi(w')$. Da dann für alle $f \in \mc{P}(V)$
\[
 \varphi^*(f)(w) = f(\varphi(w)) = f(\varphi(w')) = \varphi^*(f)(w')
\]
ist $g(w) = g(w')$ für alle $g \in \Img \varphi^*$. Aus der obigen Beobachtung folgt damit, dass $\varphi^*$ nicht surjektiv ist.


\subsection{}
Wir betrachten die Abbildung
\[
 \varphi : \R \to \R, x \mapsto x^3.
\]
Diese ist offenbar polynomiell und bijektiv. Die induzierte Abbildung
\[
 \varphi^* : \mc{P}(\R) \to \mc{P}(\R), f \mapsto \left( x \mapsto f\left(x^3\right) \right)
\]
ist aber offenbar nicht surjektiv.


























\end{document}
