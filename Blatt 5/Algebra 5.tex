\documentclass[a4paper,10pt]{article}
%\documentclass[a4paper,10pt]{scrartcl}

\usepackage{xltxtra}
\usepackage{../mystyle}

\setromanfont[Mapping=tex-text]{Linux Libertine O}
% \setsansfont[Mapping=tex-text]{DejaVu Sans}
% \setmonofont[Mapping=tex-text]{DejaVu Sans Mono}

\title{\sc Algebra I \\ \Large Blatt 5}
\author{Thorben Kastenholz \\ Jendrik Stelzner}
\date{\today}

\begin{document}
\maketitle





\section{}


\subsection{}
Für $f,g : V \to k$ ist für alle $v \in V$
\[
 (fg)(v) = 0 \Leftrightarrow f(v)g(v) = 0 \Leftrightarrow f(v) = 0 \text{ oder } g(v) = 0.
\]
Insbesondere ist daher für alle $f,g \in \mc{P}(V)$
\begin{align*}
 \mc{V}(fg)
 &= \{v \in V \mid (fg)(v) = 0\} \\
 &= \{v \in V \mid f(v) = 0 \text{ oder } g(v) = 0\} \\
 &= \{v \in V \mid f(v) = 0\} \cup \{v \in V \mid g(v) = 0\} \\
 &= \mc{V}(f) \cup \mc{V}(g).
\end{align*}


\subsection{}
Es ist
\begin{align*}
 F
 &= \left(X^2 - 9\right)^2 + \left(Y^2 - 16\right)^2 + 2\left(X^2 + 9\right)\left(Y^2 - 16\right) \\
 &= \left( \left(X^2 + 9\right) + \left(Y^2 - 16\right) \right)^2 - 36X^2 \\
 &= \left(X^2 + 9 + Y^2 - 16 + 6X\right)(X^2 + 9 + Y^2 - 16 - 6X) \\
 &= \left( (X+3)^2 + Y^2 - 16 \right) \left( (X-3)^2 + Y^2 - 16 \right).
\end{align*}
Es ist daher nach dem vorherigen Aufgabenteil
\[
 \mc{V}(F) = \mc{V}\left((X+3)^2 + Y^2 - 16\right) \cup \mc{V}\left((X-3)^2 + Y^2 - 16\right)
\]
die Vereinigung zweier Kreise, siehe Abbildung \ref{fig: Vereinigung zweier Kreise}.
\begin{figure}\centering
 \begin{tikzpicture}[scale = 0.6]
  \draw[->, very thick, black] (-8.5,0) -- (8.5,0);
  \draw[->, very thick, black] (0,-5.5) -- (0,5.5);
  \foreach \x in {-8, -6, -4, -2, 2, 4, 6, 8}
   \draw (\x, 0.2) -- (\x, -0.2) node[below] {$\x$};
  \foreach \y in {-4, -2, 2, 4}
   \draw (0.2, \y) -- (-0.2,\y) node[left] {$\y$};
  \draw[thick] (-3,0) circle (4);
  \draw[thick] (3,0) circle (4);
 \end{tikzpicture}
 \caption{$\mc{V}(F)$ aus Aufgabe 1 (b).}
 \label{fig: Vereinigung zweier Kreise}
\end{figure}


\subsection{}
Wir bemerken, dass
\[
 X^2 \left( 1 - X^2 \right) - Y^2
 = -X^4 + X^2 - Y^2
 = -\left( \left(X^2 - \frac{1}{2} \right)^2 + Y^2 - \frac{1}{4} \right).
\]
Definieren wir
\[
 f : \R^2 \to \R^2, (x,y) \mapsto \left(x^2,y\right),
\]
und den Kreis
\[
 K := \left\{ (x,y) \in \R^2 \,\middle|\, \left(x - \frac{1}{2} \right)^2 + y^2 - \frac{1}{4} = 0 \right\}
\]
so ist daher
\begin{align*}
 \mc{V}\left( X^2 \left( 1 - X^2 \right) - Y^2 \right)
 &= \left\{ (x,y) \in \R^2 \,\middle|\, \left(x^2 - \frac{1}{2} \right)^2 + y^2 - \frac{1}{4} = 0 \right\} \\
 &= f^{-1}(K).
\end{align*}
Es ergibt sich daher das Bild wie in Abbildung \ref{fig: lemis carte}
\begin{figure}\centering
 \begin{tikzpicture}[scale=1.5, domain=-1:1]
  \draw[->, very thick, black] (-1.5,0) -- (1.5,0);
  \draw[->, very thick, black] (0,-1.5) -- (0,1.5);
  \foreach \x in {-1,1}
   \draw (\x, 0.1) -- (\x, -0.1) node[below] {$\x$};
  \foreach \y in {-1,1}
   \draw (0.1, \y) -- (-0.1,\y) node[left] {$\y$};
  \draw[thick,smooth,samples=1000] plot[id=oben] function {sqrt(x**2 - x**4)};
  \draw[thick,smooth,samples=1000] plot[id=unten] function {-sqrt(x**2 - x**4)};
 \end{tikzpicture}
 \caption{$\mc{V}\left(X^2 \left( 1 - X^2 \right) - Y^2\right)$ aus Aufgabe 1 (c).}
 \label{fig: lemis carte}
\end{figure}





\section{}


\subsection{}
Es ist bekannt, dass $\varphi*$ ein $k$-Algebrahomorphismus ist. Daher ist $\varphi^*$ genau dann injektiv, wenn $\ker \varphi^* = 0$. Da für alle $f \in \mc{P}(V)$
\begin{align*}
 f \in \ker \varphi^*
 &\Leftrightarrow \varphi^*(f) = 0 \\
 &\Leftrightarrow f \circ \varphi = 0 \\
 &\Leftrightarrow (f \circ \varphi)(w) = 0 \text{ für alle } w \in W \\
 &\Leftrightarrow f(y) = 0 \text{ für alle } y \in \varphi(W)
\end{align*}
ist $\ker \varphi^* = 0$ genau dann, wenn
\[
 f_{|\varphi(W)} = 0 \Leftrightarrow f = 0.
\]
Diese Äquivalenz gilt genau dann, wenn $\varphi(W)$ Zariski-dicht in $V$ liegt.


\subsection{}
Wir bemerken zunächst, dass es für alle $w, w' \in W$ eine polynomielle Funktion $f \in \mc{P}(W)$ gibt mit $f(w) \neq f(w')$. Denn wählen wir eine Basis $w_1, \ldots, w_n$ von $W$ (möglich, da $W$ endlichdimensional ist), so können wir $w$ und $w'$ eindeutig als
\[
 w = \sum_{i=1}^n \lambda_i w_i \quad \text{und} \quad w' = \sum_{i=1}^n \lambda'_i w_i
\]
schreiben. Da $w \neq w'$ gibt es $1 \leq j \leq n$ mit $\lambda_j \neq \lambda'_j$. Für
\[
 \pi : W \to k, \sum_{i=1}^n \mu_i w_i \mapsto \mu_j
\]
ist offenbar $\pi \in \mc{P}(W)$, und da $\lambda_j \neq \lambda'_j$ ist $\pi(w) \neq \pi(w')$.

Ist $\varphi$ nicht injektiv, so gibt es $w, w' \in W$ mit $\varphi(w) = \varphi(w')$. Da dann für alle $f \in \mc{P}(V)$
\[
 \varphi^*(f)(w) = f(\varphi(w)) = f(\varphi(w')) = \varphi^*(f)(w')
\]
ist $g(w) = g(w')$ für alle $g \in \Img \varphi^*$. Aus der obigen Beobachtung folgt damit, dass $\varphi^*$ nicht surjektiv ist.


\subsection{}
Wir betrachten die Abbildung
\[
 \varphi : \R \to \R, x \mapsto x^3.
\]
Diese ist offenbar polynomiell und bijektiv. Die induzierte Abbildung
\[
 \varphi^* : \mc{P}(\R) \to \mc{P}(\R), f \mapsto \left( x \mapsto f\left(x^3\right) \right)
\]
ist aber offenbar nicht surjektiv.





\section{}
Sei $f \in \mc{P}(V)$ mit $f_{|\varphi(X)} = 0$. Dann ist $f \circ \varphi = \varphi^*(f) \in \mc{P}(W)$ mit $(f \circ \varphi)_{|X} = 0$. Da $X$ Zariski-dicht in $Y$ liegt. ist daher bereits $(f \circ \varphi)_{|Y} = 0$. Also ist $f_{|\varphi(Y)} = 0$.





\section{}
Wir definieren $\Xi : k^n \to M_n(k)$ durch
\[
 \Xi(a_1, \ldots, a_n) = 
 \begin{pmatrix}
  0 &        &        &   & a_n     \\
  1 &      0 &        &   & a_{n-1} \\
    & \ddots & \ddots &   & \vdots  \\
    &        &      1 & 0 & a_2     \\
    &        &        & 1 & a_1
 \end{pmatrix}
\]
und setzen
\[
 Z := \left\{ \Xi(a_1, \ldots, a_n) \mid a_1, \ldots, a_n \in k \right\}.
\]
Insbesondere ist
\begin{equation}\label{eq: X als Orbit von Z}  
 X = \left\{ SAS^{-1} \,\middle|\, A \in Z, S \in \GL_n(k) \right\}.
\end{equation}


\addtocounter{subsection}{1}
\subsection{}
Angenommen, es ist $A \in X$. Dann gibt es $S \in \GL_n(k)$ und $a_1, \ldots, a_n \in k$ mit
\[
 A = S \Xi(a_1, \ldots, a_n) S^{-1}.
\]
Setzen wir $v = Se_1$, wobei $e_1, \ldots, e_n$ die Standardbasis von $k^n$ bezeichnet, so ergibt sich induktiv, dass $A^{k-1} v = S e_k$ für alle $1 \leq k \leq n$. Da $S \in \GL_n(k)$ und $e_1, \ldots, e_n$ eine Basis von $k^n$ ist, ist daher auch $v, Av, \ldots, A^{n-1}v$ eine Basis von $k^n$.

Gibt es andererseits $v \in k^n$, so dass $v, Av, \ldots, A^{n-1}v$ linear unabhängig sind, so ist, da $\dim k^n = n$, dies bereits eine Basis von $k^n$. Schreiben wir
\[
 A^n v = \sum_{i=1}^n \lambda_i A^{i-1}v,
\]
so ist $A$ bezüglich der Basiswechselmatrix $S \in \GL_n(k)$, deren Spalten $v, \ldots, A^{n-1}v$ sind, von der Form
\[
 A = S \Xi(\lambda_n, \ldots, \lambda_1) S^{-1}.
\]
Also ist $A \in X$.


\subsection{}
Es ist klar, dass $h \in \mc{P}(M_n(k) \times k^n)$, und dass $h \neq 0$. Aus der Vorlesung ist bekannt, dass $Y$ deshalb Zariski-dicht in $M_n(k) \times k^n$ liegt.

Die Projektion $\pi : M_n(k) \times k^n \to M_n(k)$ ist linear und deshalb insbesondere polynomiell. Deshalb liegt, wie aus Aufgabe 3 bekannt, $\pi(Y)$ Zariski-dicht in $\pi(M_n(k) \times k^n) = M_n(k)$.

Dabei ist für alle $A \in M_n(k)$ genau dann $A \in \pi(Y)$, wenn es ein $v \in k^n$ gibt, so dass $\det(v, A_v, \ldots, A^{n-1}v) \neq 0$, also $v, Av, \ldots, A^{n-1}v$ linear unabhängin sind. Daher ist $\pi(Y) = X$ nach Aufgabenteil (b).


\subsection{}
Zunächst zeigen wir per Induktion über $n \geq 1$, dass für alle $a_1, \ldots, a_n \in k$
\[
 P_{\Xi(a_1, \ldots, a_n)}(t) = t^n - \sum_{i=1}^n a_t t^{n-i}.
\]

Für $n=1$ ist die Aussage klar. Sei daher $n \geq 2$ und es gelte die Aussage für $n-1$. Entwicklung nach der ersten Zeile ergibt dann
\begin{align*}
  &\, P_{\Xi(a_1, \ldots, a_n)}(t)
 = \det
 \begin{pmatrix}
   t &        &        &    &  -a_n     \\
  -1 &      t &        &    &  -a_{n-1} \\
     & \ddots & \ddots &    &   \vdots  \\
     &        &     -1 &  t &  -a_2     \\
     &        &        & -1 & t-a_1
 \end{pmatrix} \\
 =&\, t
  \begin{pmatrix}
   t &        &        &    &  -a_{n-1} \\
  -1 &      t &        &    &  -a_{n-2} \\
     & \ddots & \ddots &    &   \vdots  \\
     &        &     -1 &  t &  -a_2     \\
     &        &        & -1 & t-a_1
 \end{pmatrix}
 - (-1)^{n-1} a_n
 \underbrace{\begin{pmatrix}
  -1 &  t &        &        &    \\
     & -1 & t      &        &    \\
     &    & \ddots & \ddots &    \\
     &    &        &     -1 &  t \\
     &    &        &        & -1
 \end{pmatrix}}_{\in M_{n-1}(k)} \\
 =&\, t P_{\Xi(a_1, \ldots, a_{n-1})}(t) - a_n
 \underset{\text{IV}}= t\left( t^{n-1} - \sum_{i=1}^{n-1} a_i t^{n-1-i} \right) - a_n \\
 =&\, t^n - \sum_{i=1}^{n-1} a_i t^{n-i} - a_n
 = t^n - \sum_{i=1}^n a_i t^{n-i}.
\end{align*}

Setzen wir $A := \Xi(a_1, \ldots, a_n)$ für $a_1, \ldots, a_n \in k$, so ist
\[
 t^n - \sum_{i=1}^n a_i t^{n-i} = P_A(t) = t^n + \sum_{i=1}^n (-1)^i s_i(A) t^{n-i},
\]
und durch Koeffizientenvergleich ergibt sich, dass
\[
 a_i = (-1)^{n+i} s_i(A) \text{ für alle } 1 \leq i \leq n.
\]


\subsection{}
Sei $f \in \mc{P}(M_n(k))^{\GL_n(k)}$. Da $f \in \mc{P}(M_n(k))$ gibt es $\tilde{p} \in k[X_{11}, \ldots, X_{nn}]$, so dass $f$ bezüglich der Basis $(E_{ij})_{1 \leq i,j \leq n}$ von $M_n(k)$ die Form
\[
 f\left( \sum_{i,j=1}^n a_{ij} E_{ij} \right) = \tilde{p}(a_{11}, \ldots, a_{nn}) \text{ für alle } (a_{ij})_{1 \leq i,j \leq n} \in M_n(k)
\]
hat. Insbesondere gibt es daher $p' \in k[X_1, \ldots, X_n]$, so dass
\[
 f\left( \Xi(a_1, \ldots, a_n) \right) = p'(a_1, \ldots, a_n) \text{ für alle } a_1, \ldots, a_n \in k..
\]
Für alle $A = \Xi(a_1, \ldots, a_n) \in Z$ ist
\[
 a_i = (-1)^{i+1} s_i(A)
\]
und daher
\[
 f(A) = p'(a_1, \ldots, a_n) = p'\left(s_1, -s_2, \ldots, (-1)^{n+1} s_n\right)(A).
\]
Für $p \in k[X_1, \ldots, X_n]$ mit
\[
 p(X_1, \ldots, X_n) = p'(X_1, -X_2, X_3, \ldots, (-1)^{n+1} X_n)
\]
ist daher
\[
 f(A) = p(s_1, \ldots, s_n)(A) \text{ für alle } A \in Z.
\]
Da $f$ und $p(s_1, \ldots, s_n)$ invariant unter der Konjugationswirkung von $\GL_n(k)$ sind, ist wegen \eqref{eq: X als Orbit von Z} auch
\[
 f(A) = p(s_1, \ldots, s_n)(A) \text{ für alle } A \in X.
\]
Da $X$ Zariski-dicht in $M_n(k)$ liegt, ist daher bereits
\[
 f(A) = p(s_1, \ldots, s_n)(A) \text{ für alle } A \in M_n(k).
\]
Also ist
\[
 f = p(s_1, \ldots, s_n) \in k[s_1, \ldots, s_n].
\]





\end{document}
