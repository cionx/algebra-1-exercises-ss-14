\documentclass[a4paper,10pt]{article}
%\documentclass[a4paper,10pt]{scrartcl}

\usepackage{xltxtra}
\usepackage{../mystyle}

\setromanfont[Mapping=tex-text]{Linux Libertine O}
% \setsansfont[Mapping=tex-text]{DejaVu Sans}
% \setmonofont[Mapping=tex-text]{DejaVu Sans Mono}

\title{\sc Algebra I \\ \Large Blatt 3}
\author{Thorben Kastenholz \\ Jendrik Stelzner}
\date{\today}

\begin{document}
\maketitle





\section{}


\subsection{}
Das Polynom
\[
 f(x,y) := (x+iy)^{2014} (x-iy)^{2014} = \left(x^2+y^2\right)^{2014}
\]
ist symmetrisch, da $f(x,y) = f(y,x)$.

Das Polynom
\[
 g(x,y) := 4x^3 + 3y^4
\]
ist nicht symmetrisch, da $g(x,y) \neq g(y,x)$.

Das Polynom
\begin{align*}
 h(x,y)
 &:= x^5 + 3x^4y + 3x^3y^2 + x^2y^3 + x^3y^2 + 3x^2y^3 + 3xy^4 + y^5 \\
 &= x^5 + 3x^4y + 4x^3y^2 + 4x^2y^3 + 3xy^4 + y^5
\end{align*}
ist symmetrisch, da $h(x,y) = h(y,x)$.


\subsection{}
Es ist
\begin{align*}
 f(x,y)
 &= \left(x^2+y^2\right)^{2014}
 = \left((x+y)^2-2xy\right)^{2014}
 = \left(e_1^2-2e_2\right)^{2014}
\shortintertext{und}
 h(x,y)
 &= x^5 + 3x^4y + 3x^3y^2 + x^2y^3 + x^3y^2 + 3x^2y^3 + 3xy^4 + y^5 \\
 &= \left(x^2+y^2\right)\left(x^3+3x^2y+3xy^2+y^3\right) \\
 &= \left((x+y)^2-2xy\right)(x+y)^3
 = \left(e_1^2-2e_2\right)e_1^3.
\end{align*}


\subsection{}
Wir zeigen, dass $\C[x,y]^{S_2} = \C[e_1, e_2]$. Da $e_1$ und $e_2$ symmetrisch sind, und daher $e_1, e_2 \in \C[x,y]^{S_2}$,  ist klar, dass $\C[e_1, e_2] \subseteq \C[x,y]^{S_2}$.

Um zu zeigen, dass $\C[x,y]^{S_2} \subseteq \C[e_1, e_2]$, zeigen wir zunächst per Induktion über $n \in \N$, dass $x^n + y^n \in \C[e_1, e_2]$. Für $n=0$ und $n=1$ ist dies Aussage klar. Es sei daher $n \geq 2$ und es gelte die Aussage für $n-1$ und $n-2$, d.h. es gelte \mbox{$x^{n-1}+y^{n-1}, x^{n-2}+y^{n-2} \in \C[e_1, e_2]$}. Dann ist auch
\begin{align*}
 x^n + y^n
 &= \left(x^{n-1}+y^{n-1}\right)(x+y) - xy\left(x^{n-2}+y^{n-2}\right) \\
 &= e_1\left(x^{n-1}+y^{n-1}\right) - e_2\left(x^{n-2}+y^{n-2}\right)
 \in \C[e_1, e_2].
\end{align*}

Sei nun $f \in \C[x,y]^{S_2}$. Wir schreiben $f$ als
\[
 f(x,y) = \sum_{n,m \geq 0} a_{nm} x^n y^m
\]
mit $a_{nm} \in \C$ für alle $n,m \in \N$ und $a_{nm} = 0$ für fast alle $n,m \in \N$. Das $f$ symmetrisch ist, bedeutet, dass $a_{nm} = a_{mn}$ für alle $n,m \in \N$. Daher ist
\begin{align*}
 f(x,y)
 &= \sum_{n,m \geq 0} a_{nm} x^n y^m
 = \sum_{n \geq 0} a_{nn} x^n y^n + \sum_{\substack{n,m \geq 0 \\ n \neq m}} a_{nm} x^n y^m \\
 &= \sum_{n \geq 0} a_{nn} x^n y^n + \sum_{n > m \geq 0} a_{nm}\left(x^n y^m + x^m y^n\right) \\
 &= \sum_{n \geq 0} a_{nn} (xy)^n + \sum_{n > m \geq 0} a_{nm} (xy)^m \left(x^{n-m} + y^{n-m}\right) \\
 &= \sum_{n \geq 0} a_{nn} e_2^n + \sum_{n > m \geq 0} a_{nm} e_2^m \left(x^{n-m} + y^{n-m}\right)
 \in \C[e_1, e_2].
\end{align*}
Das zeigt, dass $\C[x,y]^{S_2} \subseteq \C[e_1, e_2]$.




\section{}


\subsection{}
Wir schreiben $S_2 = \{e,s\}$ mit $s^2 = e$. Wir definieren $v_1, v_2 \in \C S_2$ als
\[
 v_1 := e+s \text{ und } v_2 := e - s.
\]
Es ist klar, dass $\{v_1, v_2\}$ eine $\C$-Basis von $\C S_2$ ist, dass wir also eine Zerlegung
\begin{equation}\label{eq: Zerlegung CS_2}
 \C S_2 = \C v_1 \oplus \C v_2
\end{equation}
von $\C$-Vektorräumen haben. Wir bemerken, dass $\C v_1$ und $\C v_2$ bereits Unterdarstellungen von $\C S_2$ sind, da
\[
 e.v_1 = v_1, e.v_2 = v_2 \text{ und } s.v_1 = v_1, s.v_2 = -v_2.
\]
(Da $S_2$ linear auf $\C S_2$ wirkt genügt es die entsprechende Abgeschlossenheit unter Gruppenwirkung auf einer Basis nachzurechnen.) Da $\C v_1$ und $\C v_2$ eindimensionale Darstellungen sind, sind $\C v_1$ und $\C v_2$ irreduzibel, und damit insbesondere auch unzerlegbar. Es ist also \eqref{eq: Zerlegung CS_2} eine Zerlegung in unzerlegbare Unterdarstellungen, die alle irreduzibel sind.


\subsection{}
Sei $n \in \N, n \geq 1$ beliebig aber fest. Für $G = \Z/n\Z$ zerlegen wir die Darstellung $\C G$ von $G$ in die direkte Summe von $n$ irreduzibeln Unterdarstellungen $U_1, \ldots, U_n$.

Sei hierfür $w \in \C$ eine primitive $n$-te Einheitswurzel. Schreiben wir $g_k := k + n\Z$ für $k \in \Z$, so definieren wir $v_1, \ldots, v_n \in \C G$ durch $v_j := \sum_{k=0}^{n-1} w^{(j-1)k}$ für alle $1 \leq j \leq n$, also
\begin{align*}
 v_1 &= g_0 + g_1 + g_2 + \ldots + g_{n-1}, \\
 v_2 &= g_0 + w g_1 + w^2 g_2 + \ldots + w^{n-1} g_{n-1}, \\
 v_3 &= g_0 + w^2 g_1 + w^4 g_2 + \ldots + w^{2(n-1)} g_{n-1} \\
 v_4 &= g_0 + w^3 g_1 + w^9 g_2 + \ldots + w^{3(n-1)} g_{n-1} \\
     &\vdots \\
 v_n &= g_0 + w^{n-1} g_1 + w^{2(n-1)} g_2 + \ldots + w^{(n-1)(n-1)} g_{n-1}.
\end{align*}
Da $w$ eine primitive $n$-te Einheitswurzel ist, sind $1, w, w^2, \ldots, w^{n-1}$ paarweise verschieden. Da Elemente $v_1, \ldots, v_n$ sind daher linear unabhängig, denn es ist
\begin{align*}
 &\, \det
 \begin{pmatrix}
       1 &       1 &          1 & \hdots &              1 &          1     \\
       1 &       w &        w^2 & \hdots &        w^{n-2} &    w^{n-1}     \\
       1 &     w^2 &        w^4 & \hdots &     w^{2(n-2)} & w^{2(n-1)}     \\
  \vdots &  \vdots &     \vdots & \ddots &         \vdots &      \vdots    \\
       1 & w^{n-2} & w^{2(n-2)} & \hdots & w^{(n-2)(n-2)} & w^{(n-1)(n-2)} \\
       1 & w^{n-1} & w^{2(n-1)} & \hdots & w^{(n-2)(n-1)} & w^{(n-1)(n-1)} \\
 \end{pmatrix} \\
 =&\, \prod_{1 \leq i < j \leq n} (w^{j-1} - w^{i-1})
 = \prod_{0 \leq i < j \leq n-1} (w^j - w^i)
 \neq 0.
\end{align*}
Zur Bestimmung der Determinante beachte man, dass es sich bei der Matrix um eine Vandermonde-Matrix handelt.

Für alle $1 \leq i \leq n$ setzen wir $U_i := \C v_i$. Da $v_1, \ldots, v_n$ linear unabhängig sind erhalten wir eine Zerlegung
\begin{equation}\label{eq: C ZnZ Zerlegung}
 \C G = U_1 \oplus \ldots \oplus U_n
\end{equation}
von $\C$-Vektorräumen. Wir bemerken, dass $U_1, \ldots, U_n$ Unterdarstellungen von $\C G$ sind, denn es ist für jedes $1 \leq j \leq n$
\begin{align*}
 g_1 . v_j
 &= g_1 . \sum_{k=0}^{n-1} w^{(j-1)k} g_k
 = \sum_{k=0}^{n-1} w^{(j-1)k} g_{k+1} \\
 &= \sum_{k=0}^n w^{(j-1)(k-1)} g_k \\
 &= \sum_{k=1}^n w^{(j-1)(k-1)} g_k + w^{(j-1)(n-1)} g_n \\
 &= w^{-(j-1)} \sum_{k=1}^n w^{(j-1)k} g_k + w^{-(j-1)} g_0 \\
 &= w^{-(j-1)} \sum_{k=0}^n w^{(j-1)k} g_k
 = w^{-(j-1)} v_j
 \in U_j.
\end{align*}
Dabei nutzen wir, dass $w^{n-1} = w^n w^{-1} = w^{-1}$ und $g_n = g_0$. Da $G$ von $g_1$ erzeugt wird, zeigt dies bereits, dass $U_j$ eine Unterdarstellung von $\C G$ ist.

Da die Darstellungen $U_j$ für alel $1 \leq j \leq n$ eindimensional ist, ist $U_j$ für alle $1 \leq j \leq n$ irreduzibel, und damit insbesondere unzerlegbar. Daher ist \eqref{eq: C ZnZ Zerlegung} bereits eine Zerlegegung in unzerlegbare Unterdarstellungen, die alle irreduzibel sind.

Zuletzt merken wir noch an, dass der vorherige Aufgabenteil nur eine Sonderfall von diesem ist, da $S_2 \cong Z/2\Z$.


\subsection{}
Wir setzen $G := \Z/3\Z$, und für $n \in \Z$ setzen wir $g_n := n+3\Z \in G$. Wir bemerken, dass
\[
 U := \gen{g_0 + g_1 + g_2}_{\F_3}
\]
eine Unterdarstellung von $\F_3 G$, denn $G$ wird von $g_1$ erzeugt, und
\[
 g_1.(g_0 + g_1 + g_2)
 = g_1 + g_2 + g_3
 = g_1 + g_2 + g_0
 \in U.
\]
Da $0 \neq U \neq \F_3 G$ zeigt dies, dass $\F_3 G$ reduzibel ist.

$\F_3 G$ ist jedoch unzerlegbar: Angenommen, $\F_3 G$ wäre zerlegbar. Dann hat $\F_3 G$ nicht-triviale Unterdarstellungen $U_1, U_2$ mit $\F_3 G = U_1 \oplus U_2$.

Die lineare Wirkung von $g_1$ lässt sich bezüglich der Basis $\{g_0, g_1, g_2\}$ von $\F_3 G$ als Matrix
\[
 A := \vect{0 & 0 & 1 \\ 1 & 0 & 0 \\ 0 & 1 & 0}
\]
schreiben. Das charakteristische Polynom $\chi_A \in \F_3[X]$ von $A$ ist
\[
 \chi_A = (-t)^3 + 1 = -\left(t^3-1\right) = -(t-1)^3,
\]
da $\kchar \F_3 = 3$. Da $U_1$ und $U_2$ Unterdarstellungen von $\C G$ sind, sind $U_1$ und $U_2$ invariant unter $A$. Bezeichnet $A_{|U_1}$ die Einschränkung der linearen Wirkung von $g_1$ auf $U_1$, und analog $A_{|U_2}$ die Einschränkung auf $U_2$, so ist daher
\[
 \chi_A = \chi_{A|U_1} \cdot \chi_{A|U_2}.
\]
Da $\chi_A = -(t-1)^3$ und $\dim_{\F_3} U_1, \dim_{\F_3} U_2 \geq 1$ muss es in $U_1$ und $U_2$ daher Eigenvektoren von $A$ zum Eigenwert $1$ geben. Da $U_1 \cap U_2 = \emptyset$ müssen diese linear unabhängig sein. Daher muss der Eigenraum von $A$ zum Eigenwert $1$ mindestens zweidimensional sein.

Durch kurzes Nachrechnen ergibt sich jedoch, dass
\[
 \ker (A-I) = \gen{g_0 + g_1 + g_2}_{\F_3} = U
\]
nur eindimensional ist. Dieser Widerspruch zeigt, dass $\F_3 G$ unzerlegbar ist.



\section{}


\subsection{}
Für alle $1 \leq i \leq n$ sei $\xi_i \in \End(k[x_1, \ldots, x_n])$ definiert als $\xi_i(p) := x_i \cdot p$ für alle $p \in k[x_1, \ldots, x_n]$ die Multiplikation mit $x_i$, und $\zeta_i \in \End(k[x_1, \ldots, x_n])$ definiert als $\zeta_i(p) := \del p / \del x_i$ für alle $p \in k[x_1, \ldots, x_n]$ die formale Ableitung nach $x_i$.

Nach der universellen Eigenschaft der freien Algebra gibt es einen eindeutigen $K$-Algebrahomomorphismus $\psi : k\gen{X_1, \ldots, X_n, \del_1, \ldots, \del_n} \to \End(V)$, so dass $\psi(X_i) = \xi_i$ und $\psi(\del_i) = \zeta_i$ für alle $1 \leq i \leq n$.

Wir bemerken, nun, dass $I \subseteq \ker \psi$: Es ist klar, dass die Endomorphismen $\xi_i$ und $\xi_j$ für alle $1 \leq i, j \leq n$ miteinander kommutieren, und daher $X_i X_j - X_j X_i \in \ker \psi$ für alle $1 \leq i,j \leq n$. Genau so ist es auch klar, dass $\del_i \del_j - \del_j \del_i \in \ker \psi$ für alle $1 \leq i,j \leq n$ und $\del_i X_j - X_j \del_i \in \ker \psi$ für alle $1 \leq i,j \leq n$ mit $i \leq j$. Es ist allerdings $\del_j X_j - X_j \del_j \in \ker \psi$ für kein $1 \leq i \leq n$, denn für alle
\[
 p
 a= \sum_{i_1, \ldots, i_n \geq 0} \lambda_{i_1, \ldots, i_n} x_1^{i_1} \cdots x_n^{i_n} \in k[x_1, \ldots, x_n]
\]
ist für alle $1 \leq j \leq n$
\begin{align*}
  &\,\psi(\del_j X_j - X_j \del_j)(p) 
 =   \zeta_j(\xi_j(p)) - \xi_j(\zeta_j(p)) \\
 =&\, \sum_{i_1, \ldots, i_n \geq 0} (i_j+1) \lambda_{i_1, \ldots, i_n} x_1^{i_1} \cdots x_n^{i_n}
 - \sum_{i_1, \ldots, i_n \geq 0} i_j \lambda_{i_1, \ldots, i_n} x_1^{i_1} \cdots x_n^{i_n} \\
 =&\, \sum_{i_1, \ldots, i_n \geq 0} \lambda_{i_1, \ldots, i_n} x_1^{i_1} \cdots x_n^{i_n}
 = p
 = \psi(1)(p)
\end{align*}
Zusammen mit $\del_i X_j - X_j \del_i \in \ker \psi$ für alle $1 \leq i, j \leq n$ mit $i \neq j$ zeigt dies, dass
\[
 \del_i Xj - X_j \del_i - \delta_{ij} 1 \in \ker \psi \text{ für alle } 1 \leq i,j \leq n.
\]

Da $\ker \psi$ ein beidseitiges Ideal ist, ist damit $I \subseteq \ker \psi$. Nach dem Homomorphisatz faktorisiert $\psi$ als eindeutiger $K$-Algebrahomomorphismus
\[
 \varphi : k\gen{X_1, \ldots, X_n, \del_1, \ldots, \del_n}/I \to \End(k[x_1, \ldots, x_n]).
\]
Dieser entspricht bekanntermaßen einer $\mc{A}$-Modulstruktur auf $k[x_1, \ldots, x_n]$ via
\[
 a \cdot p = \varphi(a)(p) \text{ für alle } a \in \mc{A}, p \in k[x_1, \ldots, x_n].
\]
Nach der Konstruktion von $\varphi$ ist dabei
\begin{equation}\label{eq: A-Modulstruktur}
 X_i \cdot p = \xi_i(p) \text{ und } \del_i \cdot p = \zeta_i(z)
 \text{ für alle } 1 \leq i \leq n, p \in k[x_1, \ldots, x_n].
\end{equation}
(Wir arbeiten hier etwas unsauber, indem wir nicht zwischen das Element $X_i \in k\gen{X_1, \ldots, X_n, \del_1, \ldots, \del_n}$ und die entsprechende Restklasse $\overline{X_i} \in \mc{A}$ gleich notieren.)

Es ist klar, dass die von $\varphi$ induzierte $\mc{A}$-Modulstruktur auf $k[x_1, \ldots, x_n]$ die eindeutige $\mc{A}$-Modulstruktur auf $k[x_1, \ldots, x_n]$ ist, die \eqref{eq: A-Modulstruktur} erfüllt, denn $\mc{A}$ wird als $K$-Algebra von den Elementen $X_1, \ldots, X_n, \del_1, \ldots, \del_n$ erzeugt, und die Wirkung dieser Elemente auf $k[x_1, \ldots, x_n]$ ist durch \eqref{eq: A-Modulstruktur} eindeutig bestimmt.


\subsection{}
Es ist klar, dass die Elemente der Form
\[
 a_1 \cdots a_r \in \mc{A} \text{ mit }r \in \N \text{ und } a_i \in \{X_1, \ldots, X_n, \del_1, \ldots, \del_n\} \text{ für alle } 1 \leq i \leq r
\]
ein Erzeugendensystem von $\mc{A}$ als $k$-Vektorraum sind. Es genügt daher zu zeigen, dass sich jedes solche Element als $k$-Linearkombination von den Elementen $X^\alpha \del^\beta$ mit $\alpha, \beta \in \N^n$ schreiben lässt. Wir zeigen dies per Induktion über $r$.

Für $r = 0$ und $r = 1$ ist die Aussage klar. Es sei daher $r \geq 2$ und es gelte die Aussage für $r-1$. Es seien $a_1, \ldots, a_r \in \{X_1, \ldots, X_n, \del_1, \ldots, \del_n\}$ beliebig aber fest. Ist $a_i \in \{\del_1, \ldots, \del_n\}$ für alle $1 \leq i \leq r$ so kommutieren die $a_i$ miteinander und die Aussage ist klar.

Ist $a_1 = X_j$ für ein $1 \leq j \leq n$ so können wir $a_2 \cdots a_r$ nach Induktionsvoraussetzung als
\[
 a_2 \cdots a_n = \sum_{\alpha, \beta \in \N^n} \lambda_{\alpha, \beta} X^\alpha \del^\beta
\]
schreiben. Da die $X_i$ miteinander kommutieren ist dann auch
\[
 a_1 \cdots a_n
 = Xj \sum_{\alpha, \beta \in \N^n} \lambda_{\alpha, \beta} X^\alpha \del^\beta
 = \sum_{\alpha, \beta \in \N^n} \lambda_{\alpha, \beta} X_j X^\alpha \del^\beta
\]
eine entsprechende $k$-Linearkombination.

Ist keiner der beiden oberen Fälle erfüllt, so setzen wir
\[
 s := \min \{1 \leq i \leq r : a_i \in \{X_1, \ldots, X_n\}\}.
\]
und $1 \leq j \leq n$, so dass $a_s = X_j$. Wir betrachten den Ausdruck
\[
 a_1 \cdots a_s.
\]
Da $a_1, \ldots, a_{s-1} \in \{\del_1, \ldots, \del_n\}$ können wir
\[
 a_1 \cdots a_s = \del^\beta X_j
\]
mit $\beta = (\beta_1, \ldots, \beta_n) \in \N^n$ schreiben. Ist $\beta_j = 0$, so folgt aus $\del_i X_j = X_j \del_i$ für $i \neq j$ weiter, dass
\[
 a_1 \cdots a_r = = \del^\beta X_j a_{s+1} \cdots a_r = X_j \del^\beta a_{s+1} \cdots a_r.
\]
Damit befinden wir uns dann in einem der vorherigen Fälle.

Ist $\beta_j \neq 0$, so bemerken wir, dass sich aus $\del_j X_j = X_j \del_j + 1$ induktiv ergibt, dass
\[
 \del_j^n X_j = X_j \del_j^n + n \del_j^{n-1} \text{ für alle } n \geq 1,
\]
denn für $n=1$ gilt die Aussage, und wenn die Aussage für $n$ gilt, so ist
\begin{align*}
 \del_j^{n+1} X_j
 &= \del_j \left(\del_j^n X_j\right)
 = \del_j \left(X_j\del_j^n + n\del_j^{n-1}\right)
 = \del_j X_j \del_j^n + n \del_j^n \\
 &= (X_j \del_j + 1) \del_j^n + n \del_j^n
 = X_j \del_j^{n+1} + (n+1)\del_j^n.
\end{align*}
Da $\del_i X_j = X_j \del_i$ für $i \neq j$ ist daher
\begin{align*}
 a_1 \cdots a_s
 =&\; \del^\beta X_j
 =    \del_1^{\beta_1} \cdots \del_n^{\beta_n} X_j \\
 =&\; \del_1^{\beta_1} \cdots \del_{j-1}^{\beta_{j-1}} \left( X_j \del_j^{\beta_j} + \beta_j \del_j^{\beta_j-1}\right) \del_{j+1}^{\beta_{j+1}} \cdots \del_n^{\beta_n} \\
 =&\; X_j \del^\beta + \beta_j \del_1^{\beta_1} \cdots \del_{j-1}^{\beta_{j-1}} \del_{j}^{\beta_j-1} \del_{j+1}^{\beta_{j+1}} \cdots \del_n^{\beta_n} \\
 =&\; X_j \del^\beta + \beta_j \del^{\beta'}
\end{align*}
mit
\[
 \beta' = (\beta_1, \ldots, \beta_{j-1}, \beta_j-1, \beta_{j+1}, \ldots, \beta_n)
\]
Daher ist
\begin{align*}
 a_1 \cdots a_r
 &= \left( X_j \del^\beta + \beta_j \del^{\beta'} \right) a_{s+1} \cdots a_r \\
 &= X_j \del^\beta a_{s+1} \cdots a_r + \beta_j \del^{\beta'} a_{s+1} \cdots a_r.
\end{align*}
Der erste Summand lässt sich nach einem der vorherigen Fälle als eine entsprechende $k$-Linearkombination schreiben, und der zweite nach Induktionsvoraussetzung.

Damit haben wir gezeigt, dass die Elemente der Form $X^\alpha \del^\beta$ ein $k$-Erzeu\-gen\-den\-sys\-tem von $\mc{A}$ sind. Die Elemente sind auch linear unabhängig: Angenommen, es gebe $\lambda_1, \ldots, \lambda_r \in k \smallsetminus \{0\}$ und $\alpha^1, \ldots, \alpha^r, \beta^1, \ldots, \beta^r \in \N^n$, so dass $X^{\alpha^1} \del^{\beta^1}, \ldots, X^{\alpha^r} \del^{\beta^r}$ paarweise verschieden sind, mit 
\[
 \sum_{i=1}^r \lambda_i X^{\alpha^i} \del^{\beta^i} = 0.
\]
Sei dann $\beta^* \in \{\beta^1, \ldots, \beta^r\}$, so dass $|\beta^*|$ minimal ist. Wir betrachten das Polynom
\[
 p := x^{\beta^*} = x_1^{\beta^*_1} \cdots x_n^{\beta^*_n} \in k[x_1, \ldots, x_n].
\]
Wegen der Minimalität von $|\beta^*|$ muss für alle $\beta \in \{\beta^1, \ldots \beta^r\}$ entweder $\beta = \beta^*$ oder $\beta_i > \beta^*_i$ für ein $1 \leq i \leq n$, und damit $\del^\beta \cdot p = 0$. Für
\[
 I = \{1 \leq i \leq r : \beta^i = \beta^*\} \neq \emptyset
\]
muss also bereits
\[
 0
 = 0 \cdot p
 = \sum_{i=1}^r \lambda_i X^{\alpha^i} \del^{\beta^i} \cdot p
 = \sum_{i \in I} \lambda_i X^{\alpha^i} \del^{\beta^*} \cdot p,
 = \sum_{i \in I} \lambda_i \beta^*! X^{\alpha^i},
\]
wobei wir die Notation $\beta^*! = \beta_1! \cdots \beta_n!$ verwenden. Da die Elemente $X^{\alpha^i} \del^{\beta^i}$ für $i \in I$ paarweise verschieden sind, aber $\beta_i = \beta^*$ für alle $i \in I$, müssen die Elemente $\alpha_i$ für $i \in I$ paarweise verschieden sein. Also sind die Elemente $X^{\alpha^i}$ mit $i \in I$ in $k[x_1, \ldots, x_n]$ linear unabhängig. Daher muss $\lambda_i \beta^*! = 0$ für alle $i \in I$, wegen $\kchar k \neq 0$ also $\lambda_i \neq 0$ für alle $i \in I$. Da $I \neq \emptyset$ ist das ein Widerspruch dazu, dass $\lambda_1, \ldots, \lambda_r \neq 0$.


































\end{document}
