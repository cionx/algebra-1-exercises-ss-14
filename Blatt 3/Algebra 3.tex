\documentclass[a4paper,10pt]{article}
%\documentclass[a4paper,10pt]{scrartcl}

\usepackage{xltxtra}
\usepackage{../mystyle}

\setromanfont[Mapping=tex-text]{Linux Libertine O}
% \setsansfont[Mapping=tex-text]{DejaVu Sans}
% \setmonofont[Mapping=tex-text]{DejaVu Sans Mono}

\title{\sc Algebra I \\ \Large Blatt 3}
\author{Thorben Kastenholz \\ Jendrik Stelzner}
\date{\today}

\begin{document}
\maketitle





\section{}


\subsection{}
Das Polynom
\[
 f(x,y) := (x+iy)^{2014} (x-iy)^{2014} = \left(x^2+y^2\right)^{2014}
\]
ist symmetrisch, da $f(x,y) = f(y,x)$.

Das Polynom
\[
 g(x,y) := 4x^3 + 3y^4
\]
ist nicht symmetrisch, da $g(x,y) \neq g(y,x)$.

Das Polynom
\begin{align*}
 h(x,y)
 &:= x^5 + 3x^4y + 3x^3y^2 + x^2y^3 + x^3y^2 + 3x^2y^3 + 3xy^4 + y^5 \\
 &= x^5 + 3x^4y + 4x^3y^2 + 4x^2y^3 + 3xy^4 + y^5
\end{align*}
ist symmetrisch, da $h(x,y) = h(y,x)$.


\subsection{}
Es ist
\begin{align*}
 f(x,y)
 &= \left(x^2+y^2\right)^{2014}
 = \left((x+y)^2-2xy\right)^{2014}
 = \left(e_1^2-2e_2\right)^{2014}
\shortintertext{und}
 h(x,y)
 &= x^5 + 3x^4y + 3x^3y^2 + x^2y^3 + x^3y^2 + 3x^2y^3 + 3xy^4 + y^5 \\
 &= \left(x^2+y^2\right)\left(x^3+3x^2y+3xy^2+y^3\right) \\
 &= \left((x+y)^2-2xy\right)(x+y)^3
 = \left(e_1^2-2e_2\right)e_1^3.
\end{align*}


\subsection{}
Wir zeigen, dass $\C[x,y]^{S_2} = \C[e_1, e_2]$. Da $e_1$ und $e_2$ symmetrisch sind, und daher $e_1, e_2 \in \C[x,y]^{S_2}$,  ist klar, dass $\C[e_1, e_2] \subseteq \C[x,y]^{S_2}$.

Um zu zeigen, dass $\C[x,y]^{S_2} \subseteq \C[e_1, e_2]$, zeigen wir zunächst per Induktion über $n \in \N$, dass $x^n + y^n \in \C[e_1, e_2]$. Für $n=0$ und $n=1$ ist dies Aussage klar. Es sei daher $n \geq 2$ und es gelte die Aussage für $n-1$ und $n-2$, d.h. es gelte \mbox{$x^{n-1}+y^{n-1}, x^{n-2}+y^{n-2} \in \C[e_1, e_2]$}. Dann ist auch
\begin{align*}
 x^n + y^n
 &= \left(x^{n-1}+y^{n-1}\right)(x+y) - xy\left(x^{n-2}+y^{n-2}\right) \\
 &= e_1\left(x^{n-1}+y^{n-1}\right) - e_2\left(x^{n-2}+y^{n-2}\right)
 \in \C[e_1, e_2].
\end{align*}

Sei nun $f \in \C[x,y]^{S_2}$. Wir schreiben $f$ als
\[
 f(x,y) = \sum_{n,m \geq 0} a_{nm} x^n y^m
\]
mit $a_{nm} \in \C$ für alle $n,m \in \N$ und $a_{nm} = 0$ für fast alle $n,m \in \N$. Das $f$ symmetrisch ist, bedeutet, dass $a_{nm} = a_{mn}$ für alle $n,m \in \N$. Daher ist
\begin{align*}
 f(x,y)
 &= \sum_{n,m \geq 0} a_{nm} x^n y^m
 = \sum_{n \geq 0} a_{nn} x^n y^n + \sum_{\substack{n,m \geq 0 \\ n \neq m}} a_{nm} x^n y^m \\
 &= \sum_{n \geq 0} a_{nn} x^n y^n + \sum_{n > m \geq 0} a_{nm}\left(x^n y^m + x^m y^n\right) \\
 &= \sum_{n \geq 0} a_{nn} (xy)^n + \sum_{n > m \geq 0} a_{nm} (xy)^m \left(x^{n-m} + y^{n-m}\right) \\
 &= \sum_{n \geq 0} a_{nn} e_2^n + \sum_{n > m \geq 0} a_{nm} e_2^m \left(x^{n-m} + y^{n-m}\right)
 \in \C[e_1, e_2].
\end{align*}
Das zeigt, dass $\C[x,y]^{S_2} \subseteq \C[e_1, e_2]$.















\end{document}
