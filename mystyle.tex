% die verschiedenen benutzen Pakete
\usepackage{amsmath}
\usepackage{amssymb}
\usepackage{amsthm}
\usepackage{mathtools}
\usepackage{stmaryrd}  %für \mapsfrom
\usepackage{extarrows} %für \xlongequal
\usepackage{tikz}      %für Diagrame
\usepackage{enumerate}

% die Umgebungen mit kursiver Schrift
\newcounter{saetze}
\newtheorem{lem}[saetze]{Lemma}
\newtheorem{lem}[saetze]{Bemerkung}

\newcommand{\N}{\mathbb{N}}
\newcommand{\Z}{\mathbb{Z}}
\newcommand{\Q}{\mathbb{Q}}
\newcommand{\R}{\mathbb{R}}
\newcommand{\C}{\mathbb{C}}
\newcommand{\F}{\mathbb{F}}

\newcommand{\mc}{\mathcal}

\newcommand{\Maps}{\operatorname{Maps}}
\newcommand{\Hom}{\operatorname{Hom}}
\newcommand{\GL}{\operatorname{GL}}
\newcommand{\supp}{\operatorname{supp}}

\newcommand{\id}{\operatorname{id}}
\newcommand{\can}{\operatorname{can}}

\newcommand{\dotcup}{\ensuremath{\mathaccent\cdot\cup}}
\newcommand{\bigdotcup}{\charfusion[\mathop]{\bigcup}{\cdot}}
\newcommand{\vspan}{\operatorname{span}}
\newcommand{\kchar}{\operatorname{char}}

\newcommand{\vect}[1]{\begin{pmatrix}#1\end{pmatrix}}


% technischer Kram für \bigdotcup
\makeatletter
\def\moverlay{\mathpalette\mov@rlay}
\def\mov@rlay#1#2{\leavevmode\vtop{%
   \baselineskip\z@skip \lineskiplimit-\maxdimen
   \ialign{\hfil$\m@th#1##$\hfil\cr#2\crcr}}}
\newcommand{\charfusion}[3][\mathord]{
    #1{\ifx#1\mathop\vphantom{#2}\fi
        \mathpalette\mov@rlay{#2\cr#3}
      }
    \ifx#1\mathop\expandafter\displaylimits\fi}
\makeatother