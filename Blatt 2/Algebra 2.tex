\documentclass[a4paper,10pt]{article}
%\documentclass[a4paper,10pt]{scrartcl}

\usepackage{xltxtra}
\usepackage{../mystyle}

\setromanfont[Mapping=tex-text]{Linux Libertine O}
% \setsansfont[Mapping=tex-text]{DejaVu Sans}
% \setmonofont[Mapping=tex-text]{DejaVu Sans Mono}

\title{\sc Algebra I \\ \Large Blatt 2}
\author{Thorben Kastenholz \\ Jendrik Stelzner}
\date{\today}

\begin{document}
\maketitle





\section{}
Wir gehen im Folgenden davon aus, dass $kG$-Moduln als unitär verstanden werden, da die Aussage sonst offenbar nicht stimmt.

Es sei $\pi : G \times V \to V$ eine lineare Gruppenwirkung auf $V$. Diese entspricht einem Gruppenhomomorphismus $\tilde{\pi} : G \to \GL(V), g \mapsto \pi_g$ mit $\pi_g : v \mapsto g.v$. Wir können diesen zu einer Abbildung $\bar{\pi} : G \to \End(V), g \mapsto \pi_g$ ergänzen. Da der zugrundelegende $k$-Vektorraum von $kG$ der freie $k$-Vektorraum über $G$ ist, lässt sich $\bar{\pi}$ durch die universelle Eigenschaft des freien Vektorraums zu einer linearen Abbildung $\tau : kG \to \End(V)$ ergänzen, d.h. für alle $\sum_{g \in G} a_g g \in kG$ ist
\[
 \tau\left( \sum_{g\in G} a_g g\right) = \sum_{g \in G} a_g \bar{\pi}(g) = \sum_{g \in G} a_g \pi_g.
\]
Da $G$ eine $k$-Basis von $kG$ ist, und $\tau$ auf dieser Basis multiplikativ ist (denn $\tau_{|G} = \bar{\pi}$), ist $\tau$ auch ein Ringhomomorphismus, d.h. für alle $\sum_{g \in G} a_g g, \sum_{h \in G} b_h h \in kG$ ist
\begin{align*}
 &\, \tau\left( \left(\sum_{g \in G} a_g g\right) \cdot \left(\sum_{h \in G} b_h h\right) \right)
 = \tau\left( \sum_{g, h \in G} a_g b_h g h \right) \\
 =&\, \sum_{g,h \in G} a_g b_h \pi_{g h}
 = \sum_{g,h \in G} a_g b_h \pi_g \pi_h
 = \left(\sum_{g \in G} a_g \pi_g\right) \left(\sum_{h \in G} b_h \pi_h\right) \\
 =&\, \tau\left(\sum_{g \in G} a_g g\right) \tau\left(\sum_{h \in G} b_h h\right).
\end{align*}
Da auch $\tau(1_{kG}) = \tau(e) = \pi_e = 1_{\End(V)}$ ist $\tau : kG \to \End(V)$ ein unitaler $k$-Algebrahomomorphismus. Bekanntermaßen entspricht $\tau$ einer $kG$-Modulstruktur auf $V$ via
\begin{align*}
 \left(\sum_{g \in G} a_g g\right) \cdot v
 &:= \tau\left(\sum_{g \in G} a_g g\right)(v)
 = \left(\sum_{g \in G} a_g \pi_g\right)(v) \\
 &= \sum_{g \in G} a_g \pi_g(v)
 = \sum_{g \in G} a_g (g.v).
\end{align*}

Andererseits entspricht eine $kG$-Modulstruktur auf $V$ einem unitären $k$-Algebra\-homo\-morphismus $\Phi : kG \to \End(V), x \mapsto (v \mapsto x \cdot v)$. Insbesondere ist $\Phi$ ein unitärer Ringhomomorphismus, und induziert daher einen Gruppenhomomorphismus der Einheitengruppen
\[
 \tilde{\phi} : (kG)^\times \to (\End(V))^\times = \GL(V).
\]
Da $G \subseteq (kG)^\times$ eine Unterguppe ist (denn $g$ hat in $kG$ das Inverse $g^{-1}$) beschränkt sich $\tilde{\phi}$ zu einem Gruppenhomomorphismus $\phi : G \to \GL(V)$. $\phi$ entspricht einer linearen $G$-Gruppenwirkung auf $V$ via $g.v = \phi(g)(v)$ für alle $g \in G, v \in V$.

Die beiden Konstruktionen sind invers zueinander: Es sei $\pi : G \times V \to V$ eine lineare Gruppenwirkung auf $V$, $\tilde{\tau} : kG \to \End(V)$ der entsprechende $k$-Algebra\-homo\-morphismus, wie oben konstruiert, und $\pi' : G \to \GL(V)$ der Gruppenhomomorphismus, der wie oben durch Einschränkung von $\tau$ auf $G$ entsteht. Da für alle $g \in G, v \in V$
\[
 \pi'(g)(v) = \tau(g)(v) = \pi_g(v) = g.v
\]
ist die lineare Gruppenaktion, die $\pi'$ entspricht, genau $\pi$.

Ist andererseits $\Phi : kG \to \End(V)$ ein unitärer $k$-Algebra\-homo\-morhismus, $\pi : G \to \GL(V)$ der wie oben beschriebene, durch Einschränkung entstehende Gruppenhomomorphismus, und $\Psi : kG \to \End(V)$ der aus $\pi$ entstehende $k$-Algebra\-homo\-morphismus. Es ist klar, dass $\Phi$ und $\Psi$ auf $G \subseteq kG$ übereinstimmen. Da $G$ eine $k$-Basis von $kG$ ist, ist daher $\Phi = \Psi$.





\section{}
Es bezeichnen $\pi : G \times V \to V$ und $\tau : G \times W \to W$ die entsprechenden $G$-Wirkungen, sowie für alle $g \in G$
\[
 \pi_g : V \to V, g \mapsto g.v \text{ und }
 \tau_g : W \to W, g \mapsto g.w.
\]
Da $\pi$ und $\tau$ lineare Gruppenwirkungen sind, sind $\pi_g$ und $\tau_g$ $k$-linear für alle $g \in G$.


\subsection{}
Die gewöhnliche $G$-Wirkung auf $\Maps(W,V)$ ist definiert als
\[
 g.f = \pi_g \circ f \circ \tau_{g^{-1}}. \text{ für alle } f \in \Maps(W,V), g \in G.
\]
$\Hom_k(W,V)$ ist unter dieser Gruppenaktion abgeschlossen, da für jede $k$-lineare Abbildung $f : W \to V$ und alle $g \in G$ auch $\pi_g \circ f \circ \tau_{g^{-1}} : W \to V$ $k$-linear ist. Also induziert die $G$-Wirkung auf $\Maps(W,V)$ eine $G$-Wirkung
\[
 \sigma : G \times \Hom_k(W,V) \to \Hom_k(W,V).
\]
Da für alle $g \in G$ die Abbildung
\[
 \sigma_g : \Hom_k(W,V) \to \Hom_k(W,V), f \mapsto \pi_g \circ f \circ \tau_{g^{-1}}
\]
$k$-linear in $f$ ist, wirkt $\sigma$ linear auf $\Hom_k(W,V)$. Das zeigt, dass $\Hom_k(W,V)$ vermöge $\sigma$ eine Darstellung von $G$ ist.

Im Falle $V = k$ hat $\sigma$ die Form
\[
 \sigma_g : W^* \to W^*, f \mapsto f \circ \tau_{g^{-1}}.
\]
Dies entspricht offenbar genau der dualen Darstellung von $G$.


\subsection{}
Da $V$ und $W$ endlichdimensional sind, ist
\[
 \varphi: V \otimes_k W^* \to \Hom_k(W,V), v \otimes \lambda \mapsto (w \mapsto \lambda(w) \cdot v)
\]
bekanntermaßen ein Isomorphismus von $k$-Vektorräumen. $\varphi$ ist auch $G$-äquivariant, da für alle $g \in G, v \in V, \lambda \in W^*, w \in W$
\begin{align*}
 \varphi(g.(v \otimes \lambda))(w)
 &= \varphi((g.v) \otimes (g.\lambda))(w)
 = (g.\lambda)(w) \cdot (g.v) \\
 &= \lambda(g^{-1}.w) \cdot (g.v)
\shortintertext{und}
 (g.\varphi(v \otimes \lambda))(w)
 &= (\pi_g \circ \varphi(v \otimes \lambda) \circ \tau_{g^{-1}})(w) \\
 &= g.(\varphi(v \otimes \lambda)(g^{-1}.(w))) \\
 &= g.(\lambda(g^{-1}.w) \cdot v)
 = \lambda(g^{-1}.w) \cdot (g.v),
\end{align*}
also $\varphi(g.(v \otimes \lambda)) = g.\varphi(v \otimes \lambda)$ für alle $g \in G, v \in V, \lambda \in W^*$, und deshalb \mbox{$\varphi(g.x) = g.\varphi(x)$} für alle $g \in G, x \in V \otimes_k W^*$. (Die Elementartensoren $v \otimes \lambda$ mit $v \in V$ und $\lambda \in W^*$ sind ein Erzeugendensystem von $V \otimes_k W^*$, wegen der Linearität von $g.\varphi(x)$ und $\varphi(g.x)$ in $x$ genügt es daher die Gleichheit der beiden Funktionen für Elementartensoren zu überprüfen.)


\subsection{}

\subsubsection{}
Es ist klar, dass die Abbildung
\[
 \varphi : k \to \End_k(V), x \mapsto x \id_V
\]
$k$-linear ist. Sie ist auch $g$-äquivariant, da für alle $g \in G, \lambda \in k, v \in V$
\begin{align*}
 (g.\varphi(\lambda))(v)
 &= (\pi_g \circ \varphi(\lambda) \circ \pi_{g^{-1}})(v)
 = g.(\varphi(\lambda)(g^{-1}.v))
 = g.(\lambda(g^{-1}.v)) \\
 &= \lambda(g.g^{-1}.v)
 = \lambda v
 = \varphi(\lambda)(v)
 = \varphi(g.\lambda)(v),
\end{align*}
also $g.\varphi(\lambda) = \varphi(g.\lambda)$ für alle $g \in G, \lambda \in k$.

\subsubsection{}
Da die Abbildung $V \times V^* \to k, (v,\lambda) \mapsto \lambda(v)$ offenbar $k$-bilinear ist, induziert sie eine lineare Abbildung
\[
 \varphi : V \otimes V^* \to k, v \otimes \lambda \mapsto \lambda(v).
\]
Diese ist $g$-äquivariant, da für alle $g \in G, v \in V, \lambda \in V^*$
\begin{align*}
 \varphi(g.(v \otimes \lambda))
 &= \varphi((g.v) \otimes (g.\lambda))
 = (g.\lambda)(g.v) \\
 &= \lambda(g^{-1}.g.v) 
 = \lambda(v)
 = g.\lambda(v)
 = g.\varphi(v \otimes \lambda).
\end{align*}

Der Isomorphismus $V \otimes V^* \cong \End_k(V)$ ist durch
\[
 f: V \otimes V^* \to \End_k(V), v \otimes \lambda \mapsto (w \mapsto \lambda(w) \cdot v)
\]
gegeben. Dieser induziert eine eindeutige Abbildung $\psi : \End_k(V) \to k$, so dass das Diagramm in Abbildung \ref{fig: induziert} kommutiert.
\begin{figure}\centering
 \begin{tikzpicture}[node distance = 6em, auto]
  \node (k) {$k$};
  \node (Tensor) [above left of = k] {$V \otimes V^*$};
  \node (End) [above right of = k] {$\End_k(V)$};
  \draw[->] (Tensor) to node {$f$} (End);
  \draw[->] (Tensor) to node [swap] {$\varphi$} (k);
  \draw[->,dashed] (End) to node {$\exists!\psi$} (k);
 \end{tikzpicture}
 \caption{Die induzierte Abbildung $\psi$.}
 \label{fig: induziert}
\end{figure}
Offenbar ist $\psi = \varphi f^{-1}$.

Es sei $v_1, \ldots, v_n$ eine Basis von $V$ und $v_1^*, \ldots, v_n^*$ die entsprechende duale Basis von $V^*$. Dann ist $(v_i \otimes v_j^*)_{1 \leq i,j \leq n}$ eine Basis von $V \otimes V^*$, und $(E_{ij})_{1 \leq i,j \leq n}$ eine Basis von $\End_k(V)$, wobei
\[
 E_{ij}(v_k) = \delta_{jk} v_i =
 \begin{cases}
  v_i & \text{ falls } k=j, \\
    0 & \text{ sonst}.
 \end{cases}
\]
Da für alle $1 \leq i,j,k \leq n$
\[
 f(v_i \otimes v_j^*)(v_k)
 = v_j^*(v_k)\cdot v_i
 = \delta_{jk} v_i
 = E_{ij}(v_k)
\]
ist $f(v_i \otimes v_j^*) = E_{ij}$ für alle $1 \leq i,j \leq n$. Für $A \in \End_k(V)$ mit $A = \sum_{i,j=1}^n a_{ij} E_{ij}$ ist daher
\begin{align*}
 \psi(A)
 &= \psi\left( \sum_{i,j=1}^n a_{ij} E_{ij} \right)
 = \sum_{i,j=1}^n a_{ij} \psi(E_{ij})
 = \sum_{i,j=1}^n a_{ij} \varphi(f^{-1}(E_{ij})) \\
 &= \sum_{i,j=1}^n a_{ij} \varphi(v_i \otimes v_j^*)
 = \sum_{i,j=1}^n a_{ij} v_j^*(v_i)
 = \sum_{i,j=1}^n a_{ij} \delta_{ij}
 = \sum_{i=1}^n a_{ii}
 = \tr(A).
\end{align*}
Es ist also $\psi = \tr$.


\subsection{}
\begin{bem}
Wir fixieren eine Gruppe $G$ und einen Körper $k$. Es bezeichne $\Rep_G^k$ die Kategorie, deren Objekte die Darstellungen von $G$ über $k$ sind, zusammen mit den Morphismen
\[
 \Hom_{\Rep_G^k}(V,W) = \Hom_G(V,W)
\]
für alle Darstellungen von $V,W$ von $G$ über $k$. Die Verknüpfung zweier Morphismen ist ihre Verknüpfung als Funktionen. (Es ist bekannt, dass $\Rep_G^k$ tatsächlich eine Kategorie ist.)

Wir bemerken zunächst, dass für $V \in \Rep_G^k$
\[
 V^G = \{v \in V : g.v = v \text{ für alle } g \in G\}
\]
eine Unterdarstellung von $G$ ist, auf der $G$ trivial wirkt.

\begin{proof}
 Sei $V \in \Rep_G^k$ beliebig aber fest. Bezeichnet $\pi : G \times V \to V$ die Gruppenwirkung auf $V$, so ist
 \begin{align*}
  V^G
  &= \bigcap_{g \in G} \{v \in V : g.v = v\}
  = \bigcap_{g \in G} \{v \in V : \pi_g(v) = v\} \\
  &= \bigcap_{g \in G} \{v \in V : (\pi_g-\id_V)(v) = 0\}
  = \bigcap_{g \in G} \ker(\pi_g - \id_V)
 \end{align*}
 ein Untervektorraum von $V$. Dass $G$ trivial auf $V^G$ wirkt ist offensichtlich.
\end{proof}

Als Nächstes bemerken wir, dass für $V,W \in \Rep_G^k$ jeder Homomorphismus von Darstellungen $f \in \Hom_G(V,W)$ durch Einschränkung einen Homomorphismus (von Darstellungen) $f^G : V^G \to W^G$ induziert.
\begin{proof}
 Für alle $v \in V^G$ ist
 \[
  g.(f(v)) = f(g.v) = f(v) \text{ für alle } g \in G,
 \]
 also $f(v) \in W^G$ für alle $v \in V^G$. Daher ist
 \[
  f^G : V^G \to W^G, v \mapsto f(v)
 \]
 eine wohldefinierte $k$-lineare Abbildung. Dass $f$ $G$-äquivariant ist, folgt direkt daraus, dass $G$ trivial auf $V^G$ und $W^G$ wirkt.
\end{proof}

Zusammengefasst ergibt dies, dass $T : \Rep_G^k \to \Rep_G^k$ mit
\begin{align*}
 T(V) &:= V^G \text{ für alle } V \in \Rep_G^k \text{ und } \\
 T(f) &:= f^G \text{ für alle } f \in \Hom_G(V,W) \text{ mit } V,W \in \Rep_G^k
\end{align*}
ein (kovarianter) Funktor ist.

\begin{proof}
  Es ist klar, dass $T(\id_V) = \id_V^V = \id_{T(V)}$ für alle $V \in \Rep_G^k$. Auch ist klar, dass $T$ mit der Komposition verträglich ist, da es sich bei $T(f)$ für $f \in \Hom_G(V,W)$ um die Einschränkung von $f$ handelt.
\end{proof}
\end{bem}

Seien nun $G$ und $k$ wieder wie in der Aufgabe. Wir wissen, dass
\[
 \Hom_k(k,V) \in \Rep_G^k \text{ und } V \in \Rep_G^k.
\]
Die Abbildung 
\[
 \varphi: \Hom_k(k,V) \to V, f \mapsto f(1)
\]
ist offenbar ein Isomorphismus von $K$-Vektorräumen. $\varphi$ ist auch $G$-äquivariant, da für alle $g \in G$ und $f \in \Hom_k(k,V)$
\[
 \varphi(g.f) = (g.f)(1) = g.f(g^{-1}.1) = g.f(1) = g.\varphi(f).
\]
Es ist also $\varphi$ ein Isomorphismus von Darstellungen. Da $T$ (definiert wie in der Bemerkung) ein Funktor ist, erhalten wir einen Isomorphismus von Darstellungen
\[
 \psi : (\Hom_k(k,V))^G \to V^G, f \mapsto f(1).
\]
Da $(\Hom_k(k,V))^G = \Hom_G(k,V)$ zeigt dies die Aussage.












\end{document}
