\documentclass[a4paper,10pt]{article}
%\documentclass[a4paper,10pt]{scrartcl}

\usepackage{xltxtra}
\usepackage{../mystyle}

\setromanfont[Mapping=tex-text]{Linux Libertine O}
% \setsansfont[Mapping=tex-text]{DejaVu Sans}
% \setmonofont[Mapping=tex-text]{DejaVu Sans Mono}

\title{\sc Algebra I \\ \Large Blatt 2}
\author{Thorben Kastenholz \\ Jendrik Stelzner}
\date{\today}

\begin{document}
\maketitle





\section{}
Wir gehen im Folgenden davon aus, dass $kG$-Moduln als unitär verstanden werden, da die Aussage sonst offenbar nicht stimmt.

Es sei $\pi : G \times V \to V$ eine lineare Gruppenwirkung auf $V$. Diese entspricht einem Gruppenhomomorphismus $\tilde{\pi} : G \to \GL(V), g \mapsto \pi_g$ mit $\pi_g : v \mapsto g.v$. Wir können diesen zu einer Abbildung $\bar{\pi} : G \to \End(V), g \mapsto \pi_g$ ergänzen. Da der zugrundelegende $k$-Vektorraum von $kG$ der freie $k$-Vektorraum über $G$ ist, lässt sich $\bar{\pi}$ durch die universelle Eigenschaft des freien Vektorraums zu einer linearen Abbildung $\tau : kG \to \End(V)$ ergänzen, d.h. für alle $\sum_{g \in G} a_g g \in kG$ ist
\[
 \tau\left( \sum_{g\in G} a_g g\right) = \sum_{g \in G} a_g \bar{\pi}(g) = \sum_{g \in G} a_g \pi_g.
\]
Da $G$ eine $k$-Basis von $kG$ ist, und $\tau$ auf dieser Basis multiplikativ ist (denn $\tau_{|G} = \bar{\pi}$), ist $\tau$ auch ein Ringhomomorphismus, d.h. für alle $\sum_{g \in G} a_g g, \sum_{h \in G} b_h h \in kG$ ist
\begin{align*}
 &\, \tau\left( \left(\sum_{g \in G} a_g g\right) \cdot \left(\sum_{h \in G} b_h h\right) \right)
 = \tau\left( \sum_{g, h \in G} a_g b_h g h \right) \\
 =&\, \sum_{g,h \in G} a_g b_h \pi_{g h}
 = \sum_{g,h \in G} a_g b_h \pi_g \pi_h
 = \left(\sum_{g \in G} a_g \pi_g\right) \left(\sum_{h \in G} b_h \pi_h\right) \\
 =&\, \tau\left(\sum_{g \in G} a_g g\right) \tau\left(\sum_{h \in G} b_h h\right).
\end{align*}
Da auch $\tau(1_{kG}) = \tau(e) = \pi_e = 1_{\End(V)}$ ist $\tau : kG \to \End(V)$ ein unitaler $k$-Algebrahomomorphismus. Bekanntermaßen entspricht $\tau$ einer $kG$-Modulstruktur auf $V$ via
\begin{align*}
 \left(\sum_{g \in G} a_g g\right) \cdot v
 &:= \tau\left(\sum_{g \in G} a_g g\right)(v)
 = \left(\sum_{g \in G} a_g \pi_g\right)(v) \\
 &= \sum_{g \in G} a_g \pi_g(v)
 = \sum_{g \in G} a_g (g.v).
\end{align*}

Andererseits entspricht eine $kG$-Modulstruktur auf $V$ einem unitären $k$-Algebra\-homo\-morphismus $\Phi : kG \to \End(V), x \mapsto (v \mapsto x \cdot v)$. Insbesondere ist $\Phi$ ein unitärer Ringhomomorphismus, und induziert daher einen Gruppenhomomorphismus der Einheitengruppen
\[
 \tilde{\phi} : (kG)^\times \to (\End(V))^\times = \GL(V).
\]
Da $G \subseteq (kG)^\times$ eine Unterguppe ist (denn $g$ hat in $kG$ das Inverse $g^{-1}$) beschränkt sich $\tilde{\phi}$ zu einem Gruppenhomomorphismus $\phi : G \to \GL(V)$. $\phi$ entspricht einer linearen $G$-Gruppenwirkung auf $V$ via $g.v = \phi(g)(v)$ für alle $g \in G, v \in V$.

Die beiden Konstruktionen sind invers zueinander: Es sei $\pi : G \times V \to V$ eine lineare Gruppenwirkung auf $V$, $\tilde{\tau} : kG \to \End(V)$ der entsprechende $k$-Algebra\-homo\-morphismus, wie oben konstruiert, und $\pi' : G \to \GL(V)$ der Gruppenhomomorphismus, der wie oben durch Einschränkung von $\tau$ auf $G$ entsteht. Da für alle $g \in G, v \in V$
\[
 \pi'(g)(v) = \tau(g)(v) = \pi_g(v) = g.v
\]
ist die lineare Gruppenaktion, die $\pi'$ entspricht, genau $\pi$.

Ist andererseits $\Phi : kG \to \End(V)$ ein unitärer $k$-Algebra\-homo\-morhismus, $\pi : G \to \GL(V)$ der wie oben beschriebene, durch Einschränkung entstehende Gruppenhomomorphismus, und $\Psi : kG \to \End(V)$ der aus $\pi$ entstehende $k$-Algebra\-homo\-morphismus. Es ist klar, dass $\Phi$ und $\Psi$ auf $G \subseteq kG$ übereinstimmen. Da $G$ eine $k$-Basis von $kG$ ist, ist daher $\Phi = \Psi$.







\end{document}
