\documentclass[a4paper,10pt]{article}
%\documentclass[a4paper,10pt]{scrartcl}

\usepackage{xltxtra}
\usepackage{../mystyle}

\setromanfont[Mapping=tex-text]{Linux Libertine O}
% \setsansfont[Mapping=tex-text]{DejaVu Sans}
% \setmonofont[Mapping=tex-text]{DejaVu Sans Mono}

\title{\sc Algebra I \\ \Large Blatt 11}
\author{Thorben Kastenholz \\ Jendrik Stelzner}
\date{\today}

\begin{document}
\maketitle





\addtocounter{section}{2}





\section{}
Wir behaupten, dass $R/\mf{m}$ bis auf Isomorphie der einzige einfache $R$-Modul ist.

Wir bemerken zunächst folgendes: Für ein Linksideal $I \subseteq R$ entsprechen die $R$-Untermoduln von $R/I$ in bijektiver Weise den Linksdealen von $R$, die $I$ enthalten via
\begin{align*}
 \{J \subseteq R \mid J \text{ ist ein Linksideal mit } I \subseteq J\} &\xleftrightarrow{1:1}\{\text{$R$-Untermoduln von $R/I$}\} \\
 J &\mapsto J/I,
\end{align*}
Daher ist $R/I$ genau dann irreduzibel als $R$-Modul, wenn $I$ ein maximales Linksideal in $R$ ist. Inbesondere ist daher $R/\mf{m}$ ein einfacher $R$-Modul.

Ist andererseits $M$ ein einfacher $R$-Modul, so gibt es $m \in M$ mit $m \neq 0$. Da $R$ unitär ist, ist $Rm \neq 0$ und wegen der Irreduziblität von $M$ somit $Rm = M$. Wir erhalten so einen $R$-Modulepimorphismus
\[
 \pi : R \to M, r \mapsto rm.
\]
$\ker \pi$ ist ein Untermodul, also Linksideal, in $R$, und da $M$ einfach ist, ist $\ker \pi$ ein maximales Linksideal. Also ist $\ker \pi = \mf{m}$. Somit ist
\[
 M \cong R/\ker \pi = R/\mf{m}.
\]













\end{document}
