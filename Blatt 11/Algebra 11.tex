\documentclass[a4paper,10pt]{article}
%\documentclass[a4paper,10pt]{scrartcl}

\usepackage{xltxtra}
\usepackage{../mystyle}

\setromanfont[Mapping=tex-text]{Linux Libertine O}
% \setsansfont[Mapping=tex-text]{DejaVu Sans}
% \setmonofont[Mapping=tex-text]{DejaVu Sans Mono}

\title{\sc Algebra I \\ \Large Blatt 11}
\author{Thorben Kastenholz \\ Jendrik Stelzner}
\date{\today}

\begin{document}
\maketitle





\addtocounter{section}{2}





\section{}
Wir behaupten, dass $R/\mf{m}$ bis auf Isomorphie der einzige einfache $R$-Modul ist.

Wir bemerken zunächst folgendes: Für ein Linksideal $I \subseteq R$ entsprechen die $R$-Untermoduln von $R/I$ in bijektiver Weise den Linksdealen von $R$, die $I$ enthalten via
\begin{align*}
 \{J \subseteq R \mid J \text{ ist ein Linksideal mit } I \subseteq J\} &\xleftrightarrow{1:1}\{\text{$R$-Untermoduln von $R/I$}\} \\
 J &\mapsto J/I,
\end{align*}
Daher ist $R/I$ genau dann irreduzibel als $R$-Modul, wenn $I$ ein maximales Linksideal in $R$ ist. Inbesondere ist daher $R/\mf{m}$ ein einfacher $R$-Modul.

Ist andererseits $M$ ein einfacher $R$-Modul, so gibt es $m \in M$ mit $m \neq 0$. Da $R$ unitär ist, ist $Rm \neq 0$ und wegen der Irreduziblität von $M$ somit $Rm = M$. Wir erhalten so einen $R$-Modulepimorphismus
\[
 \pi : R \to M, r \mapsto rm.
\]
$\ker \pi$ ist ein Untermodul, also Linksideal, in $R$, und da $M$ einfach ist, ist $\ker \pi$ ein maximales Linksideal. Also ist $\ker \pi = \mf{m}$. Somit ist
\[
 M \cong R/\ker \pi = R/\mf{m}.
\]





\section{}
Wir gehen davon aus, dass $A$ unitär ist.


\subsection*{(a) $\Rightarrow$ (b)}
Wir definieren
\[
 \varepsilon : A \to k, a \mapsto (a,1).
\]
Aus der $k$-Bilinearität von $(\cdot,\cdot)$ folgt die $k$-Linearität von $\varepsilon$. Da $(\cdot,\cdot)$ nicht entartet ist, gibt es für jedes $a \in A$ mit $a \neq 0$ ein $b \in A$ mit $(a,b) \neq 0$, also
\[
 \varepsilon(ba) = (ba,1) = (b,a) = (a,b) \neq 0.
\]


\subsection*{(b) $\Rightarrow$ (a)}
Für alle $a,b \in A$ definieren wir
\[
 (a,b) := \varepsilon(ab).
\]
Aus (b) folgt direkt, dass dies auf $A$ eine nicht entartete symmetrische Bilinearform definiert. Die Assoziativität von $(\cdot, \cdot)$ folgt direkt aus der Assoziativität der Multiplikation auf $A$.


\subsection*{(b) $\Leftrightarrow$ (c)}
Es genügt zu zeigen, dass die jeweiligen Bedingungen (ii) in (b) und in (c) äquivalent sind. Sei hierfür $\varepsilon : A \to k$ $k$-linear.

Da $A$ unitär ist, enthält $\ker \varepsilon$ genau dann ein von $0$ verschiedenes Linksideal, wenn es ein von $0$ verschiedenes Links-Hauptideal enthält. Dies gilt genau dann, wenn es $a \in A$, $a \neq 0$ gibt mit $\varepsilon(ba) = 0$ für alle $b \in A$. Dass $\varepsilon(ba) = 0$ für alle $b \in A$ ist äquivalent dazu, dass es kein $b \in A$ gibt mit $\varepsilon(ba) \neq 0$.
















\end{document}
