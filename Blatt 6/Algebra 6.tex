\documentclass[a4paper,10pt]{article}
%\documentclass[a4paper,10pt]{scrartcl}

\usepackage{xltxtra}
\usepackage{../mystyle}

\setromanfont[Mapping=tex-text]{Linux Libertine O}
% \setsansfont[Mapping=tex-text]{DejaVu Sans}
% \setmonofont[Mapping=tex-text]{DejaVu Sans Mono}

\title{\sc Algebra I \\ \Large Blatt 6}
\author{Thorben Kastenholz \\ Jendrik Stelzner}
\date{\today}

\begin{document}
\maketitle





\addtocounter{section}{1}
\section*{Aufgabe 1 und Aufgabe 4}
Im Folgenden sei $k$ ein (nicht notwendigerweise unendlicher) Körper und $V$ ein endlichdimensionaler $k$-Vektorraum. Für Teilmengen $X \subseteq V$ wollen wir untersuchen, in welchen Teilmengen von $V$ $X$ Zariski-dicht liegt, und wie sich Zariski-Dichtheit charakterisieren lässt.

Hierfür bemerken wir, dass die Teilmengen von $V$, in denen $X$ Zariski-dicht liegt, unter Vereinigung abgeschlossen sind: Ist $(U_i)_{i \in I}$ eine nichtleere Kollektion von Mengen mit $X \subseteq U_i \subseteq V$ für alle $i \in I$, so dass $X$ für alle $i \in I$ Zariski-dicht in $U_i$ liegt, so liegt $X$ auch Zariski-dicht in
\[
 U := \bigcup_{i \in I} U_i.
\]
Denn es ist $I \neq \emptyset$ und deshalb $X \subseteq U \subseteq V$, und für $f \in \mc{P}(V)$ mit $f_{|X} = 0$ ist $f_{|U_i} = 0$ für alle $i \in I$, also auch $f_{|U} = 0$.

Diese Beobachtung motiviert die folgende Definition:

\begin{defi}
 Für $X \subseteq V$ ist
 \[
  Z(X) := \bigcup \left\{ Y \mid X \subseteq Y \subseteq V \text{ und $X$ liegt Zariski-dicht in $Y$} \right\}
 \]
 der Zariski-Abschluss von $X$. $X$ heißt Zariski-abgeschlossen, wenn $Z(X) = X$.
\end{defi}


Mithilfe des Zariski-Abschlusses können wir nun den Begriff der Zariski-Dichtheit charakterisieren.


\begin{lem}\label{lem: erste Aussagen über Zariski-Abschluss}
 Sei $X \subseteq V$.
 \begin{enumerate}[a)]
  \item
  $X$ liegt genau dann Zariski-dicht in $Y \subseteq V$, wenn $X \subseteq Y \subseteq Z(X)$. Inbesondere ist $Z(X)$ die größte Teilmenge von $V$, in der $X$ Zariski-dicht liegt.
  \item
  $X$ ist genau dann Zariski-abgeschlossen, wenn $X$ in keiner echt größeren Teilmenge von $V$ Zariski-dicht liegt.
  \item
  $Z(X)$ ist die kleinste Zariski-abgeschlossene Menge, die $X$ enthält. 
 \end{enumerate}
\end{lem}
\begin{proof}
 \begin{enumerate}[a)]
  \item
  Liegt $X$ Zariski-dicht in $Y \subseteq V$, so ist ist $X \subseteq Y$, und nach der Definition von $Z(X)$ auch $Y \subseteq Z(X)$. Da die Mengen, in denen $X$ Zariski-dicht liegt, unter Vereinigung abgeschlossen sind, ist $X$ Zariski-dicht in $Z(X)$, und damit ist auch in jeder Teilmenge $Y \subseteq Z(X)$ mit $X \subseteq Y$.
  \item
  Ist $X$ Zariski-abgeschlossen, so gilt für jede Teilmenge $Y \subseteq V$, in der $X$ Zariski-dicht liegt, dass $Y \subseteq Z(X) = X$. Liegt andererseits $X$ in keiner echt größeren Teilmenge von $V$ Zariski-dicht, so ist
  \[
   \left\{ Y \mid X \subseteq Y \subseteq V \text{ und $X$ liegt Zariski-dicht in $Y$} \right\} = \{X\} 
  \]
  und somit $Z(X) = X$.
  \item
  $Z(X)$ ist Zariski-abgeschlossen, denn $X$ liegt Zariski-dicht in $Z(X)$. Für jede Teilmenge $Y \subseteq V$, in der $Z(X)$ Zariski-dicht liegt, liegt deshalb auch $X$ Zarisk-dicht, weshalb $Y \subseteq Z(X)$.
  
  Für eine Zariski-abgeschlossene Teilmenge $Y \subseteq V$ mit $X \subseteq Y \subseteq Z(X)$ ist, da $X$ Zariski-dicht in $Z(X)$ liegt, auch $Y$ Zariski-dicht in $Z(X)$. Da $Y$ Zariski-abgeschlossen ist, also in keiner echt größeren Teilmenge von $V$ Zariski-dicht liegt, ist $Z(X) = Y$. \qedhere
 \end{enumerate}
\end{proof}


Für Teilmengen $X \subseteq Y \subseteq V$ ist $X$ genau dann Zariski-dicht in $Y$, wenn für jede Funktion $f \in \mc{P}(V)$ die Einschränkung $f_{|Y}$ bereits eindeutig durch $f_{|X}$ bestimmt ist. Dies legt die Vermutung nahe, dass sich der Zariski-Abschluss und die Zariski-Abgeschlossenheit von $X \subseteq V$ mithilfe von Polynomfunktionen formulieren lassen.


\begin{lem}\label{lem: Charakterisierung durch Verschwindungsmengen}
 Sei $X \subseteq V$. Dann ist
 \[
  Z(X) = \mc{V}( \mc{I}(X) ),
 \]
 und  $X$ ist genau dann Zariski-abgeschlossen, wenn
 \[
  X = \mc{V}(\mf{a}).
 \]
 für eine Teilmenge $\mf{a} \subseteq \mc{P}(V)$.
\end{lem}
\begin{proof}
 Sei $\mf{a} \subseteq \mc{P}(V)$. Sei $Y \subseteq V$, so dass $\mc{V}(\mf{a})$ Zariski-dicht in $Y$ liegt. Für alle $f \in \mf{a}$ ist $f_{|\mc{V}(\mf{a})} = 0$, also auch $f_{|Y} = 0$. Daher ist $Y \subseteq \mc{V}(\mf{a})$. Da $\mc{V}(\mf{a})$ in keiner echt größeren Teilmenge von $V$ Zariski-dicht liegt, ist $\mc{V}(\mf{a})$ nach Lemma \ref{lem: erste Aussagen über Zariski-Abschluss} Zariski-abgeschlossen.
 
 $X$ ist Zariski-dicht in $\mc{V}(\mc{I}(X))$, da $X \subseteq \mc{V}(\mc{I}(X))$ und für alle $f \in \mc{P}(V)$
 \[
  f_{|X} = 0 \Rightarrow f \in \mc{I}(X) \Rightarrow f_{|\mc{V}(\mc{I}(X))} = 0.
 \]
 Deshalb ist
 \[
  X \subseteq \mc{V}(\mc{I}(X)) \subseteq Z(X).
 \]
 Da $\mc{V}(\mc{I}(X))$ Zariski-abgeschlossen ist, ist auch $\mc{V}(\mc{I}(X)) \supseteq Z(X)$. Also ist
 \[
  Z(X) = \mc{V}(\mc{I}(X)).
 \]
 Insbesondere ist $X$ genau dann Zariski-abgeschlossen, wenn
 \[
  X = \mc{V}(\mc{I}(X)).
 \]
\end{proof}


Wie der Begriff der Zariski-Abgeschlossenheit bereits nahelegt, lassen sich die bisherigen Beobachtungen auch topologisch formulieren. 


\begin{lem}
 Die Zariski-abgeschlossenen Teilmengen von $V$ definieren eine Topologie auf $V$, in der die abgeschlossenen Mengen genau die Zariski-abgeschlossenen Mengen sind.
\end{lem}
\begin{proof}
 $\emptyset = \mc{V}(\{1\})$ und $V = \mc{V}(\{0\})$ sind Zariski-abgeschlossen. Für eine Familien $(A_i)_{i \in I}$ von Zariski-abgeschlossenen Mengen ist nach Lemma \ref{lem: Charakterisierung durch Verschwindungsmengen} auch $\bigcap_{i \in I} A_i$ Zariski-ab\-ge\-schlos\-sen, da
 \begin{align*}
  \bigcap_{i \in I} A_i
  = \bigcap_{i \in I} \mc{V}(\mc{I}(A_i))
  = \mc{V}\left( \bigcup_{i \in I} \mc{I}(A_i) \right).
 \end{align*}
 
 Sind $A, B \subseteq V$ Zariski-abgeschlossen, so ist auch $A \cup B$ Zariski-ab\-ge\-schlos\-sen: Ist $A \cup B = V$ so ist nichts zu zeigen. Ansonsten sei $y \in V \smallsetminus (A \cup B)$ beliebig aber fest. Da $A$ Zariski-abgeschlossen ist, ist $A$ nicht Zariski-dicht in $A \cup \{y\}$. Es gibt deshalb ein $f \in \mc{P}(V)$ mit
\[
 f_{|A} = 0 \quad \text{und} \quad f(y) \neq 0
\]
Analog gibt es $g \in \mc{P}(V)$ mit
\[
 g_{|B} = 0 \quad \text{und} \quad g(y) \neq 0
\]
Für $fg \in \mc{P}(V)$ ist deshalb
\[
 (fg)_{|A \cup B} = 0 \quad \text{und} \quad (fg)(y) \neq 0
\]
Also ist $A \cup B$ nicht Zariski-dicht in $A \cup B \cup \{y\}$. Wegen der Beliebigkeit von \mbox{$y \in V \smallsetminus (A \cup B)$} zeigt dies, dass $A \cup B$ in keiner echt größeren Teilmenge von $V$ Zariski-dicht liegt, weshalb $A \cup B$ Zariski-abgeschlossen ist.

Daraus ergibt sich induktiv, dass $A_1 \cup \ldots \cup A_n$ für alle Zariski-abgeschlossenen Mengen $A_1, \ldots, A_n \subseteq V$ ebenfalls Zariski-abgeschlossen ist.
\end{proof}


\section{}
Für alle
\begin{gather*}
 A = \vect{a & b \\ c & d} \in M_2(k)
\shortintertext{ist}
 A^2 = \vect{a^2 + bc & ab + bd \\ ac + cd & bc + d^2},
\shortintertext{also}
 \tr(A) = a+d \text{ und }
 \tr_2(A) = a^2 + d^2.
\end{gather*}
Für $I,J \subseteq M_2(k)$ mit
\begin{gather*}
 I := \vect{1 & 0 \\ 0 & 1} \text{ und } J := \vect{1 & 1 \\ 1 & 1}
\shortintertext{ist also}
 \tr(I) = \tr(J) = \tr_2(I) = \tr_2(J) = 0.
\end{gather*}
 Deshalb ist $f(I) = f(J)$ für alle $f \in k[\tr,\tr_2]$. Für $\det \in \mc{P}(M_2(k))^{\GL_2(k)}$ ist jedoch
\[
 \det(I) = 1 \neq 0 = \det(J).
\]
Das zeigt, dass $\det \not\in k[\tr, \tr_2]$. Also wird $\mc{P}(M_2(k))^{\GL_2(k)}$ nicht von $\tr, \tr_2$ erzeugt.





\section{}


\subsection{}
Es sei
\[
 b_1 := p_2 - p_1 \neq 0
\]
und $b_2 \in V$, so dass $\{b_1, b_2\}$ eine $k$-Basis von $V$ ist. Ist
\begin{gather*}
 p_1 = \lambda_1 b_1 + \lambda_2 b_2,
\shortintertext{so ist}
 p_2 = (\lambda_1+1) b_1 + \lambda_2 b_2.
\end{gather*}
Für die Koordinatentransformationen
\[
 \psi_i : V \to k, \mu_1 b_1 + \mu_2 b_2 \mapsto \mu_i \qquad \text{für } i = 0, 1,
\]
so ist, wie aus der Vorlesung bekannt,
\begin{align*}
 \mc{P}(V)(\{p_1\}) &= \left( \psi_1 - \lambda_1, \psi_2 - \lambda_2 \right) \text{ und} \\
 \mc{P}(V)(\{p_2\}) &= \left( \psi_1 - \lambda_1-1, \psi_2 - \lambda_2 \right).
\end{align*}
Es ist klar, dass
\[
 \mc{I}(\{p_1, p_2\})
 = \mc{I}(\{p_1\}) \cap \mc{I}(\{p_2\})
 \supseteq \mc{I}(\{p_1\}) \cdot \mc{I}(\{p_2\}).
\]
Sei $f \in \mc{I}(\{p_1, p_2\})$ beliebig aber fest. Da $f \in \mc{I}(\{p_1\})$, also
\[
 f \in \left( \psi_1 - \lambda_1, \psi_2 - \lambda_2 \right),
\]
gibt es $r_1, r_2 \in \mc{P}(V)$ mit
\[
 f = r_1 (\psi_1 - \lambda_1) + r_2(\psi_2 - \lambda_2).
\]
Da auch $f \in \mc{I}(\{p_2\})$ ist
\[
 0 = f(p_2) = r_1(p_2) (\lambda_1 + 1 - \lambda_1) + r_2(p_2) (\lambda_2 - \lambda_2) = r_1(p_2).
\]
Daher ist $r_1 \in \mc{I}(\{p_2\})$. Es gibt also $s_1, s_2 \in \mc{P}(V)$ mit
\[
 r_1 = s_1 (\psi_1 - \lambda_1 - 1) + s_2 (\psi_2 - \lambda_2).
\]
Daher ist
\begin{align*}
 f
 &= s_1(\psi_1 - \lambda_1)(\psi_1 - \lambda_1 - 1) + s_2(\psi_1 - \lambda_1)(\psi_2 - \lambda_2) + r_2(\psi_2 - \lambda_2) \\
 &= s_1(\psi_1 - \lambda_1)(\psi_1 - \lambda_1 - 1) + \left( s_2(\psi_1 - \lambda_1) + r_2\right)(\psi_2 - \lambda_2)
\end{align*}
Da
\begin{align*}
  &\, \mc{I}(\{p_1\}) \cdot \mc{I}(\{p_2\}) \\
 =&\, \left( \psi_1 - \lambda_1, \psi_2 - \lambda_2 \right) \cdot \left( \psi_1 - \lambda_1-1, \psi_2 - \lambda_2 \right) \\
 =&\, ( (\psi_1-\lambda_1)(\psi_1-\lambda_1-1), (\psi_2-\lambda_2)(\psi_1-\lambda_1-1), \\
  &\, (\psi_1-\lambda_1)(\psi_2-\lambda_2), (\psi_2-\lambda_2)^2 ) \\
 =&\, \left( (\psi_1-\lambda_1)(\psi_1-\lambda_1-1), (\psi_2-\lambda_2) \right)
\end{align*}
ist $f \in \mc{I}(\{p_1\}) \cdot \mc{I}(\{p_2\})$. Wegen der Beliebigkeit von $f \in \mc{I}(\{p_1, p_2\})$ folgt, dass
\[
 \mc{I}(\{p_1, p_2\}) = \mc{I}(\{p_1\}) \cdot \mc{I}(\{p_2\}).
\]


\subsection{}
Es sei $k$ ein beliebiger Körper, $V = k$ und $X = Y = \{0\}$. Dann ist
\[
 \mc{I}(X \cup Y) = \mc{I}(\{0\}) = (x) \subseteq \mc{P}(V),
\]
wobei $x : k \to k, \lambda \mapsto \lambda$. Daher ist
\[
 \mc{I}(X) \cdot \mc{I}(Y) = (x) \cdot (x) = (x \cdot x) = \left(x^2\right) \neq (x) = \mc{I}(X \cup Y).
\]





























\end{document}
