\documentclass[a4paper,10pt]{article}
%\documentclass[a4paper,10pt]{scrartcl}

\usepackage{xltxtra}
\usepackage{../mystyle}

\setromanfont[Mapping=tex-text]{Linux Libertine O}
% \setsansfont[Mapping=tex-text]{DejaVu Sans}
% \setmonofont[Mapping=tex-text]{DejaVu Sans Mono}

\title{\sc Algebra I \\ \Large Blatt 6}
\author{Thorben Kastenholz \\ Jendrik Stelzner}
\date{\today}

\begin{document}
\maketitle





\section{}
$X \subseteq W$ ist genau dann die Verschwindungsmenge einer Menge von Polynomen $\mf{a} \in \mc{P}(V)$, wenn $f(x) = 0$ für alle $f \in \mf{a}, x \in X$ und es für jedes $y \in W \smallsetminus X$ ein $f \in \mf{a}$ gibt, so dass $f(y) \neq 0$. Für alle $y \in W \smallsetminus X$ ist dann $f$ nicht Zarisksi-dicht in $X \cup \{y\}$.

Ist $X$ für alle $y \in W \smallsetminus X$ nicht Zariski-dicht in $X \cup \{y\}$, so ist $X$ in keiner echt größeren Teilmenge von $W$ Zariski-dicht, denn ist $X \subsetneq Y \subseteq W$ und $X$ Zariski-dicht in $Y$, so ist $X$ auch Zariski-dicht in $X \cup \{y\}$ für alle $y \in Y \smallsetminus X$.

Ist $X$ in keiner echt größeren Teilmenge von $W$ Zariski-dicht, so betrachten wir
\[
 \mf{a} := \mc{I}(X) = \{f \in \mc{P}(V) \mid f(x) = 0 \text{ für alle } x \in X\}.
\]
Es ist klar, dass $X \subseteq \mc{V}(\mf{a})$, und da für alle $f \in \mc{P}(V)$
\[
 f_{|X} = 0
 \Rightarrow f \in \mf{a}
 \Rightarrow f_{|\mc{V}(\mf{a})} = 0
\]
liegt $X$ Zariski-dicht in $\mc{V}(\mf{a})$. Nach Annahme ist deshalb $X = \mc{V}(\mf{a})$.










\end{document}
