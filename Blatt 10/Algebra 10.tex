\documentclass[a4paper,10pt]{article}
%\documentclass[a4paper,10pt]{scrartcl}

\usepackage{xltxtra}
\usepackage{../mystyle}

\setromanfont[Mapping=tex-text]{Linux Libertine O}
% \setsansfont[Mapping=tex-text]{DejaVu Sans}
% \setmonofont[Mapping=tex-text]{DejaVu Sans Mono}

\title{\sc Algebra I \\ \Large Blatt 10}
\author{Thorben Kastenholz \\ Jendrik Stelzner}
\date{\today}

\begin{document}
\maketitle






\section{}
Es sei $\{v_1, \ldots, v_n\}$ eine $k$-Basis von $V$ und $\{w_1, \ldots, w_m\}$ eine $k$-Basis von $W$. Für eine Darstellung $X$ von $A$ schreiben wir
\begin{align*}
 &\rho_X : A \to \End_k(X), a \mapsto (x \mapsto ax)
\end{align*}


\subsection{}
Ist $V \cong W$, so gibt es einen Isomorphismus $\varphi : V \to W$ von Darstellungen von $A$. Inbesondere ist $\varphi$ $k$-linear und somit $\{\varphi(v_1), \ldots, \varphi(v_n)\}$ eine $k$-Basis von $W$. Es sei $a \in A$ beliebig aber fest. Bezeichnet $A$ die darstellende Matrix von $\rho_V(a)$ bezüglich $\{v_1, \ldots, v_n\}$, so ist dies, da $\varphi$ ein $k$-Algebrahomomorphismus ist, auch die darstellende Matrix von $\rho_W(a)$ bezüglich $\{\varphi(v_1), \ldots, \varphi(v_n)\}$. Inbesondere ist deshalb
\[
 \chi_V(a) = \tr A = \chi_W(a).
\]


\subsection{}
Es sei $a \in A$ beliebig aber fest. Da $a$ komponentenweise auf $V \oplus W$ wirkt, ist die darstellende Matrix von $\rho_{V \oplus W}$ bezüglich der $k$-Basis
\[
 \{(v_1, 0), \ldots, (v_n, 0), (0, w_1), \ldots, (0, w_m)\}
\]
von $V \oplus W$ der Form
\[
 C =
 \begin{pmatrix}
  A & 0 \\
  0 & B
 \end{pmatrix},
\]
wobei $A$ die darstellende Matrix von $\rho_V(a)$ bezüglich $\{v_1, \ldots, v_n\}$ ist, und $B$ die darstellende Matrix von $\rho_W(a)$ bezüglich der Basis $\{w_1, \ldots, w_m\}$ ist. Inbesondere ist daher
\[
 \chi_{V \oplus W}(a) = \tr C = \tr A + \tr B = \chi_V(a) + \chi_W(a).
\]


\subsection{}
Da $\rho_V(e) = \id_V$ ist die darstellende Matrix von $\rho_V(e)$ bezüglich jeder Basis von $V$ die $n \times n$-Einheitsmatrix über $k$, und somit
\[
 \chi_V(e) = n \bmod \kchar k = \dim_k(V) \bmod \kchar k.
\]


\subsection{}
Es sei $a \in A$ beliebig aber fest. Bezeichnet $A$ die darstellende Matrix von $\rho_V(a)$ bezüglich $\{v_1, \ldots, v_n\}$ und $B$ die darstellende Matrix von $\rho_W(a)$ bezüglich $\{w_1, \ldots, w_m\}$, so ist, da $\rho_{V \otimes W}(a) = \rho(V)(a) \otimes \rho(W)(a)$, die darstellende Matrix von $\rho_{V \otimes W}(a)$ bezüglich der $k$-Basis
\[
 \{v_1 \otimes w_1, v_1, \otimes w_2, \ldots, v_1 \otimes  w_m, v_2 \otimes w_1, \ldots, v_n \otimes w_m\}
\]
von $V \otimes W$ von der Form
\[
 C =
 \begin{pmatrix}
  a_{11} B & a_{12} B & \ldots & a_{1n} B \\
  a_{21} B & a_{22} B & \ldots & a_{2n} B \\
    \vdots &   \vdots & \ddots &   \vdots \\
  a_{n1} B & a_{n2} B & \ldots & a_{nn} B
 \end{pmatrix}.
\]
Daher ist
\[
 \chi_{V \otimes W}(a) = \tr C = \sum_{i=1}^n a_{ii} \tr{B} = \tr A \tr B = \chi_V(a) \chi_W(a).
\]


\subsection{}
Es sei $g \in G$ beliebig aber fest. Bezeichnet $A$ die darstellende Matrix von $\rho_V(g)$ bezüglich $\{v_1, \ldots, v_n\}$, so ist die $A^{-1}$ die darstellende Matrix von $\rho_V\left(g^{-1}\right)$ bezüglich $\{v_1, \ldots, v_n\}$ und $A^* = (A^{-1})^T$ die darstellende Matrix von $\rho_{V^*}(g)$ bezüglich $\{v_1^*, \ldots, v_n^*\}$. Deshalb ist
\[
 \chi_{V^*}(g) = \tr\left(\left(A^{-1}\right)^T\right) = \tr\left( A^{-1} \right) = \chi_V\left(g^{-1}\right).
\]





\section{}


\subsection{}
Wir betrachten $V = \R^2$. $G = \Z/3\Z$ wirke auf $V$, indem $1 \in G$ durch eine Rotation um $2\pi/3$ und $2 \in G$ eine Rotation um $4\pi/3$ (in gleicher Orientierung) wirkt. Es ist klar, dass $V$ so zu einer Darstellung von $G$ wird. Diese ist irreduzibel: Ist $U \subseteq V$ eine Unterdarstellung mit $U \neq 0$, so gibt es $v \in U$ mit $v \neq 0$. Da $v$ und $1.v$ linear unabhängig sind, ist dann bereits $U = \gen{v,1.v} = V$.





\section{}


\subsection{}
Da $kG$ als $k$-Vektorraum von $G \subseteq kG$ erzeugt wird, ist $f \in Z(kG)$ genau dann, wenn $f \cdot \chi_g = \chi_g \cdot f$ für alle $g \in G$. Dabei ist für alle $h \in G$
\[
 (f \cdot \chi_g)(h) = \sum_{y \in G} f(y) \cdot \chi_g\left(y^{-1}h\right) = f\left(hg^{-1}\right)
\]
und
\[
 (\chi_g \cdot f)(h) = \sum_{y \in G} \chi_g(y) \cdot f\left(y^{-1}h\right) = f\left(g^{-1}h\right).
\]
Mit $h' = hg^{-1}$ erhalten wir so, dass $f \in Z(kG)$ genau dann, wenn für alle $h' \in G$
\[
 f(h') = f\left(hg^{-1}\right) = f\left(g^{-1}h\right) = f\left(g^{-1}hg\right).
\]















\end{document}
