\documentclass[a4paper,10pt]{article}
%\documentclass[a4paper,10pt]{scrartcl}

\usepackage{xltxtra}
\usepackage{../mystyle}

\setromanfont[Mapping=tex-text]{Linux Libertine O}
% \setsansfont[Mapping=tex-text]{DejaVu Sans}
% \setmonofont[Mapping=tex-text]{DejaVu Sans Mono}

\title{\sc Algebra I \\ \Large Blatt 9}
\author{Thorben Kastenholz \\ Jendrik Stelzner}
\date{\today}

\begin{document}
\maketitle



\addtocounter{section}{1}





\section{}
Sei zunächst $p > 0$ prim und $m \geq 1$. Wir bemerken, dass $\Z/p^m\Z$ genau dann semisimple ist, wenn $m = 1$.

Hierfür bemerken wir zunächst, dass $\Z/p^m\Z$ unzerlegbar ist. Denn ist
\[
 \Z/p^m\Z = G_1 \oplus \ldots \oplus G_k,
\]
so muss $|G_i| \mid p^m$ für alle $i = 1, \ldots, k$, also $|G_i| = p^{m_i}$ mit $m_i \leq m$ für alle $i = 1, \ldots, k$. Für ein Element $a \in \Z/p^m\Z$ ist dann
\[
 \ord a \leq \kgV(|G_1|, \ldots, |G_k|) = p^{\max_{i=1,\ldots,k} m_i}.
\]
Da $\ord 1 = m$ für $1 \in \Z/p^m\Z$ muss $\max_{i=1,\ldots,k} m_i = m$, also $|G_i| = p^m = |\Z/p^m\Z|$ für ein $1 \leq i \leq k$, und somit bereits $\Z/p^m\Z = G_i$. 

Da $\Z/p^m\Z$ unzerlegbar ist, ist $\Z/p^m\Z$ genau dann semisimple, wenn es irreduzibel ist, also genau dann, wenn $m = 1$.


Für den allgemeinen Fall sei $n \geq 1$. Es sei $n = p_1^{\nu_1} \cdots p_k^{\nu_k}$ eine Primfaktorzerlegung von $n$ mit $p_i \neq p_j$ für $i \neq j$ und $\nu_i \geq 1$ für alle $i = 1, \ldots, k$. Da $\Z$ ein Hauptidealring ist, ist
\[
 n\Z = (n) = (\kgV(p_1^{\nu_1}, \ldots, p_k^{\nu_k})) = \bigcap_{i=1}^k (p_i^{\nu_i}).
\]
Da $\Z$ ein Hauptidealring ist, ist
\[
 (p_i^{\nu_i}) + (p_j^{\nu_j}) = (\ggT(p_i^{\nu_i},p_j^{\nu_j})) = (1) = \Z,
\]
für $i \neq j$. Die Ideale $(p_i^{\nu_i})$ sind also paarweise koprim zueinander. Nach dem chinesischen Restklassensatz gibt es also einen Isomorphismus von Ringen
\[
 \Z/(n) = \Z/\bigcap_{i=1}^k (p_i^{\nu_i}) \cong \prod_{i=1}^k \Z/(p_i^{\nu_i}),
\]
also insbesondere einen Isomorphismus von abelschen Gruppen
\[
 \Z/n\Z \cong \bigoplus_{i=1}^k \Z/(p_i^{\nu_i})\Z.
\]
$\Z/n\Z$ ist genau dann semisimple, wenn jeder dieser Summanden semisimple ist. Nach der obigen Beobachtung gilt dies genau dann, wenn $\nu_i = 1$ für alle $i = 1, \ldots, k$, wenn also $n$ quadratfrei ist.








\end{document}
