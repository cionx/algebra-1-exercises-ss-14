\documentclass[a4paper,10pt]{article}
%\documentclass[a4paper,10pt]{scrartcl}

\usepackage{xltxtra}
\usepackage{../mystyle}

\setromanfont[Mapping=tex-text]{Linux Libertine O}
% \setsansfont[Mapping=tex-text]{DejaVu Sans}
% \setmonofont[Mapping=tex-text]{DejaVu Sans Mono}

\title{\sc Algebra I \\ \Large Blatt 8}
\author{Thorben Kastenholz \\ Jendrik Stelzner}
\date{\today}

\begin{document}
\maketitle

\section{}

\begin{lem}
 Es sei $R$ ein Ring und $M$ ein $R$-Modul. Dann sind äquivalent:
 \begin{enumerate}[i)]
  \item $M$ ist noethersch, d.h. jede aufsteigende Kette von Untermoduln von $M$
  \[
   M_0 \subseteq M_1 \subseteq M_2 \subseteq \ldots
  \]
  stabilisiert.
  \item Jeder Untermodul von $M$ ist endlich erzeugt über $R$.
 \end{enumerate}
\end{lem}
\begin{proof}
 Angenommen, $M$ ist noethersch. Es sei $M' \subseteq M$ ein Untermodul. Dann definieren wir eine eine aufsteigende Folge von Untermoduln von $M'$ wie folgt: Wir beginnen mit $M_0 := 0$. Ist $M_i$ definiert und $M_i \neq M'$, so gibt es $m_{i+1} \in M' \smallsetminus M_i$, und wir setzen $M_{i+1} := M_i + R m_{i+1}$; ansonsten setzen wir $M_{i+1} := M_i = M'$. Da $M$ noethersch ist, stabilisert die aufsteigende Kette
 \[
  0 = M_0 \subsetneq M_1 \subsetneq M_2 \subsetneq M_3 \subsetneq \ldots
 \]
 von Untermoduln von $M$. Nach Konstruktion der $M_i$ gibt es daher $n \in \N$ mit
 \[
  M' = M_n = Rm_1 + \ldots + Rm_n = (m_1, \ldots, m_n).
 \]
 Das zeigt, dass $M'$ ein endlich erzeugter $R$-Modul ist.
 
 Sei andererseits jeder Untermodul von $M$ endlich erzeugt über $R$. Für eine aufsteigende Kette
 \[
  M_0 \subseteq M_1 \subseteq M_2 \subseteq \ldots
 \]
 von Untermoduln von $M$ setzen wir
 \[
  M' := \bigcup_{k \in \N} M_k.
 \]
 $M'$ ist ein Untermodul von $M$ und somit endlich erzeugt. Nach Annahme gibt es daher $m_1, \ldots, m_n \in M'$ mit
 \[
  M' = (m_1, \ldots, m_n).
 \]
 Nach Definition von $M'$ gibt es ein $N \in \N$ mit $m_1, \ldots, m_n \in M_N$. Es ist daher $M_N = M$ und somit auch $M_k = M$ für alle $k \geq N$. Also stabilisert die Kette.
\end{proof}

Insbesondere ist ein kommutiver Ring $R$ genau dann noethersch, wenn jede aufsteigende Kette von Idealen
\[
 I_0 \subseteq I_1 \subseteq I_2 \subseteq I_3 \subseteq \ldots
\]
in $R$ stabilisiert.


\subsection{}
Da $k$ kommutativ ist und nur zwei Ideale enthält, ist $k$ offenbar noethersch. Induktiv ergibt sich damit aus dem Hilbertschen Basissatz direkt, dass auch $k[x_1, \ldots, x_n]$ für alle $n \in \N$ noethersch ist.
















\end{document}
