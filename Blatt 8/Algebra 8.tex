\documentclass[a4paper,10pt]{article}
%\documentclass[a4paper,10pt]{scrartcl}

\usepackage{xltxtra}
\usepackage{../mystyle}

\setromanfont[Mapping=tex-text]{Linux Libertine O}
% \setsansfont[Mapping=tex-text]{DejaVu Sans}
% \setmonofont[Mapping=tex-text]{DejaVu Sans Mono}

\title{\sc Algebra I \\ \Large Blatt 8}
\author{Thorben Kastenholz \\ Jendrik Stelzner}
\date{\today}

\begin{document}
\maketitle

\section{}

\begin{lem}\label{lem: Äquivalente Definitionen von noethersch}
 Es sei $R$ ein Ring und $M$ ein $R$-Modul. Dann sind äquivalent:
 \begin{enumerate}[i)]
  \item $M$ ist noethersch, d.h. jede aufsteigende Kette von Untermoduln von $M$
  \[
   M_0 \subseteq M_1 \subseteq M_2 \subseteq \ldots
  \]
  stabilisiert.
  \item Jeder Untermodul von $M$ ist endlich erzeugt über $R$.
 \end{enumerate}
 Insbesondere ist ein kommutiver Ring $R$ genau dann noethersch, wenn jede aufsteigende Kette von Idealen
 \[
  I_0 \subseteq I_1 \subseteq I_2 \subseteq I_3 \subseteq \ldots
 \]
 in $R$ stabilisiert.
\end{lem}
\begin{proof}
 Angenommen, $M$ ist noethersch. Es sei $M' \subseteq M$ ein Untermodul. Dann definieren wir eine eine aufsteigende Kette von Untermoduln von $M$ wie folgt: Wir beginnen mit $M_0 := 0$. Ist $M_i$ definiert und $M_i \neq M'$, so gibt es $m_{i+1} \in M' \smallsetminus M_i$, und wir setzen $M_{i+1} := M_i + R m_{i+1}$; ansonsten setzen wir $M_{i+1} := M_i = M'$. Da $M$ noethersch ist, stabilisiert die aufsteigende Kette
 \[
  0 = M_0 \subseteq M_1 \subseteq M_2 \subseteq \ldots
 \]
 von Untermoduln von $M$. Nach Konstruktion der $M_i$ gibt es daher $n \in \N$ mit
 \[
  M' = M_n = Rm_1 + \ldots + Rm_n = (m_1, \ldots, m_n).
 \]
 Also ist $M'$ ein endlich erzeugter $R$-Modul.
 
 Sei andererseits jeder Untermodul von $M$ endlich erzeugt über $R$. Für eine aufsteigende Kette
 \[
  M_0 \subseteq M_1 \subseteq M_2 \subseteq \ldots
 \]
 von Untermoduln von $M$ setzen wir
 \[
  M' := \bigcup_{k \in \N} M_k.
 \]
 $M'$ ist ein Untermodul von $M$ und somit endlich erzeugt. Also gibt es $m_1, \ldots, m_n \in M'$ mit
 \[
  M' = (m_1, \ldots, m_n).
 \]
 Nach Definition von $M'$ gibt es ein $N \in \N$ mit $m_1, \ldots, m_n \in M_N$. Es ist daher $M_N = M$ und somit auch $M_k = M$ für alle $k \geq N$. Also stabilisiert die Kette.
\end{proof}




\subsection{}
$k$ ist kommutativ und noethersch, da $k$ nur zwei Ideale enthält. Induktiv ergibt sich damit aus dem Hilbertschen Basissatz direkt, dass auch $k[x_1, \ldots, x_n]$ für alle $n \in \N$ noethersch ist.


\subsection{}
Es sei $R$ ein kommutativer noetherscher Ring. Wir nehmen an, dass $R[X]$ nicht noethersch ist. Nach Lemma \ref{lem: Äquivalente Definitionen von noethersch} gibt es dann ein Ideal $I \subseteq R[X]$ das nicht endlich erzeugt über $R[X]$ ist. (Inbesondere ist $I \neq 0$.)

Wir definieren eine Folge $(f_i)_{i \geq 1}$ von Polynomen $f_i \in I$ wie folgt: Wir wählen $f_1 \in I \smallsetminus \{0\}$ mit minimalen Grad. Ist $f_i$ definiert, so ist, da $I$ nicht endlich erzeugt ist,
\[
 (f_1, \ldots, f_i) \neq I.
\]
Es sei dann $f_{i+1} \in I \smallsetminus (f_1, \ldots, f_i)$ vom minimalen Grad. Man bemerke, dass stets $\deg f_i \leq \deg f_{i+1}$.

Für alle $i \geq 1$ definieren wir $a_i \in R$ als den Leitkoeffizienten von $f_i$ und setzen
\[
 J_i := (a_1, \ldots, a_i) \subseteq R.
\]
Da $R$ noethersch ist, stabilisiert die aufsteigende Kette von Idealen
\[
 J_1 \subseteq J_2 \subseteq J_3 \subseteq \ldots
\]
Es gibt also ein $n \geq 1$ mit
\[
 (a_1, \ldots, a_{n+1}) = J_{n+1} = J_n = (a_1, \ldots, a_n),
\]
was äquivalent dazu ist, dass $a_{n+1} \in (a_1, \ldots, a_n)$. Es gibt also $r_1, \ldots, r_n \in R$ mit
\[
 a_{n+1} = \sum_{i=1}^n r_i a_i.
\]
Deshalb ist
\[
 g := \sum_{i=1}^n r_i f_i \cdot X^{\deg f_{n+1} - \deg f_i} \in (f_1, \ldots, f_n)
\]
ein Polynom vom gleichem Grad und mit gleichen Leitkoeffizienten wie $f_{n+1}$. Da $f_{n+1} \not\in (f_1, \ldots, f_n)$ ist auch $f_{n+1}-g \not\in (f_1, \ldots, f_n)$. Da $\deg (f_{n+1}-g) < \deg f_{n+1}$ steht dies im Widerspruch zur Gradminimimalität von $f_{n+1}$.

Es ist also jedes Ideal in $R[X]$ endlich erzeugt, und $R[X]$ somit noethersch.






\section{}


\subsection{}
Es sei
\[
 0 \to M' \xlongrightarrow{i} M \xlongrightarrow{\pi} M'' \to 0
\]
eine kurze exakte Sequenz von $R$-Moduln. Wir können o.B.d.A. davon ausgehen, dass $M' \subseteq M$ ein Untermodul ist, $i$ die kanonische Inklusion, $M'' = M/M'$ und $\pi$ die kanonische Projektion.

Angenommen, $M$ ist noethersch. Jede aufsteigende Ketten von Untermoduln
\[
 M_0 \subseteq M_1 \subseteq M_2 \subseteq \ldots
\]
von $M'$ ist auch eine aufsteigende Kette von Untermoduln von $M$ und stabilisiert somit. Also ist $M'$ noethersch. Jede aufsteigende Kette von Untermoduln
\[
 N_0 \subseteq N_1 \subseteq N_2 \subseteq \ldots
\]
von $M'' = M/M'$ liefert eine aufsteigende Kette von Untermoduln
\[
 \pi^{-1}(N_0) \subseteq \pi^{-1}(N_1) \subseteq \pi^{-1}(N_2) \subseteq \ldots
\]
von $M$, die $M'$ enthalten. Da $M$ noethersch ist stabilisiert diese Kette, und da $\pi$ surjektiv ist, und somit
\[
 \pi(\pi^{-1}(N_i)) = N_i \text{ für alle } i \in \N
\]
stabilisiert auch die Kette in $M''$. Also ist $M''$ noethersch. (Aus den Beweisen geht auch direkt hervor, dass Untermoduln und Quotientenmoduln noetherscher Moduln ebenfalls noethersch sind.)

Angenommen, $M'$ und $M''$ sind noethersch. Es sei dann
\[
 M_0 \subseteq M_1 \subseteq M_2 \subseteq \ldots
\]
eine aufsteigende Kette von Untermoduln von $M$. Für alle $i \in \N$ setzen wir
\[
 M'_i := M' \cap M_i \text{ und } M''_i := \pi(M_i)
\]
und erhalten so aufsteigende Ketten von Untermoduln von $M'$ und $M''$. Da diese noethersch sind stabilisieren dies Ketten. Es gibt also $N \in \N$ mit
\[
 M'_n = M'_N \text{ und } M''_n = M''_N \text{ für alle } n \geq N.
\]
Daher ist auch $M_n = M_N$ für alle $n \geq N$. (Denn für alle $n \geq N$ enthalten wir das folgende kommutive Diagramm mit exakten Zeilen.
\begin{center}
 \begin{tikzpicture}[auto, node distance = 5em]
 \node (01) {$0$};
 \node (M'_N) [right of = 01] {$M'_N$};
 \node (M_N) [right of = M'_N] {$M_N$};
 \node (M''_N) [right of = M_N] {$M''_N$};
 \node (02) [right of = M''_N] {$0$};
 \node (03) [below of = 01] {$0$};
 \node (M'_n) [right of = 03] {$M'_n$};
 \node (M_n) [right of = M'_n] {$M_n$};
 \node (M''_n) [right of = M_n] {$M''_n$};
 \node (04) [right of = M''_n] {$0$};
 \draw[->] (01) to (M'_N);
 \draw[->] (M'_N) to node {$i$} (M_N);
 \draw[->] (M_N) to node {$\pi$} (M''_N);
 \draw[->] (M''_N) to (02);
 \draw[double equal sign distance] (01) to (03);
 \draw[double equal sign distance] (M'_N) to (M'_n);
 \draw[left hook->] (M_N) to (M_n);
 \draw[double equal sign distance] (M''_N) to (M''_n);
 \draw[double equal sign distance] (02) to (04);
 \draw[->] (03) to (M'_n);
 \draw[->] (M'_n) to node {$i$} (M_n);
 \draw[->] (M_n) to node {$\pi$} (M''_n);
 \draw[->] (M''_n) to (04);
 \end{tikzpicture}
\end{center}
Nach dem Fünferlemma ist die Inklusion bereits eine Gleichheit.)


\subsection{}
Sind $M_1, M_2$ zwei noethersche $R$-Moduln, so ist auch $M_1 \oplus M_2$ noethersch. Dies ergibt sich aus dem vorherigen Aufgabenteil durch die kurze exakte Sequenz
\[
 0 \to M_1 \to M_1 \oplus M_2 \to M_2 \to 0.
\]
Induktiv ergibt sich daher, dass für beliebige noethersche $R$-Moduln $M_1, \ldots, M_n$ auch $M_1 \oplus \ldots \oplus M_n$ noethersch ist. Ist $R$ noethersch, und somit noethersch als Modul über sich selbst, so ist daher insbesonder $R^n$ für alle $n \in \N$ als $R$-Modul noethersch.

Für einen endlich erzeugten $R$-Modul $M$ gibt es $m_1, \ldots, m_n \in M$, so dass der Modulhomomorphismus
\[
 f: R^n \to M, (r_1, \ldots, r_n) \to \sum_{i=1}^n r_i m_i
\]
surjektiv ist. Da $R^n$ noethersch als $R$-Modul ist, und $M \cong R^n / \ker f$ ist damit auch $M$ noethersch über $R$.


\subsection{}
Es bezeichne $\pi : R \to R/I$ die kanonische Projektion. $\pi$ ist sowohl ein Ringhomomorphismus als auch ein $R$-Modulhomomorphismus, wobei für alle $r, s \in R$
\[
 r \cdot \pi(s) = \pi(rs) = \pi(r) \pi(s),
\]
also für alle $r \in R, m \in R/I$
\[
 r \cdot m = \pi(r) \cdot m.
\]

Da $R$ als $R$-Modul noethersch ist, ist auch der Quotientenmodul $R/I$ noethersch über $R$. Offenbar entsprechen die Ideale in $R/I$ genau den $R$-Untermoduln von $R/I$. Ein Ideal $J \subseteq R/I$ ist daher als $R$-Modul endlich erzeugt, es gibt also $m_1, \ldots, m_n \in J$ mit
\[
 J = \sum_{i=1}^n R m_i = \sum_{i=1}^n \pi(R) m_i = \sum_{i=1}^n (R/I) m_i.
\]
Das zeigt, dass $J$ als Ideal in $R/I$ endlich erzeugt ist. Da jedes Ideal in $R/I$ endlich erzeugt ist, ist $R/I$ als Ring noethersch.


\section{}
Wir gehen im folgenden davon aus, dass $A, B$ und $C$ kommutativ sind.

Da $C$ als $A$-Algebra endlich erzeugt ist, gibt es $x_1, \ldots, x_n \in C$ mit
\[
 C = A[x_1, \ldots, x_n],
\]
und da $C$ als $B$-Modul endlich erzeugt ist, gibt es $y_1, \ldots, y_n \in C$ mit
\[
 C = By_1 + \ldots + By_n.
\]
Inbesondere gibt es daher für alle $1 \leq i \leq n$ Koeffizienten $b_{ij} \in B, 1 \leq j \leq m$ mit
\[
 x_i = \sum_{j=1}^m b_{ij} y_j \text{ für alle } i = 1, \ldots, n,
\]
und für alle $1 \leq i,j \leq n$ Koeffizienten $b_{ijk} \in B, 1 \leq k \leq n$ mit
\[
 y_i y_j = \sum_{k=1}^n b_{ijk} y_k.
\]
Wir setzen
\[
 B_0 := A[b_{ij}, b_{ijk} \mid 1 \leq i,j,k \leq n]
\]
Da $A$ noethersch und kommutativ ist, ist auch
\[
 A[x_1, \ldots, x_{n^2 + n^3}]
\]
noethersch (dies ergibt sich analog zum Aufgabe 1 (a)), und somit nach Aufgabe 2 (c) auch $B_0$ als Quotient dieses Polynomringes. 

Wir behaupten, dass
\[
 B_0 y_1 + \ldots + B_0 y_n = C.
\]
Dass $\sum_{i=1}^n B_n y_i \subseteq C$ ist klar. Andererseits ist für alle $1 \leq r,s \leq k$
\[
 x_r x_s
 = \left( \sum_{i=1}^m b_{ri} y_i \right) \left( \sum_{j=1}^m b_{sj} y_j \right)
 = \sum_{i,j=1}^m b_{ri} b_{sj} y_i y_j
 = \sum_{i,j,k=1}^m b_{ri} b_{sj} b_{ijk} y_k,
\]
also $x_r x_s \in C$. Analog ergibt sich induktiv, dass $x_1^{\nu_1} \cdots x_n^{\nu_n} \in \sum_{i=1}^n B_0 y_i$ für alle $\nu_1, \ldots, \nu_n \in \N$. Da $C$ als $A$-Modul von diesen Elementen erzeugt wird, folgt, dass $C \subseteq \sum_{i=1}^n B_0 y_i$.

Es ist also $C$ ein endlich erzeugter $B_0$-Modul. Da $B_0$ noethersch ist, ist $C$ damit nach Aufgabe 2 (b) noethersch als $B_0$-Modul. Daher ist $B \subseteq C$ als $B_0$-Modul ebenfalls endlich erzeugt. Es gibt also $z_1, \ldots, z_s \in B$ mit
\[
 B_0 z_1 + \ldots + B_0 z_s = B.
\]
Es ist daher
\[
 A[b_{ij}, b_{ijk}, z_l \mid 1 \leq i,j,k \leq n, 1 \leq l \leq s ]
 = B_0 [z_l \mid 1 \leq l \leq s]
 = B,
\]
also $B$ als $A$-Algebra endlich erzeugt.





\section{}
Wir nehmen an, dass $K$ nicht algebraisch über $k$ sei.
Es seien $x_1, \ldots, x_m \in K$ mit
\[
 K = k[x_1, \ldots, x_m].
\]
Es sei $A \subseteq \{x_1, \ldots, x_n\}$ eine maximale Menge algebraisch unabhängiger Elemente. (Eine solche existiert, da es nur endlich viele Teilmengen gibt.) Wir können durch passende Nummerierung o.B.d.A. davon ausgehen, dass $A = \{x_1, \ldots, x_r\}$. Nach Annahme ist $r \geq 1$, da die $x_i$ sonst alle algebraisch über $k$ wären, und somit auch $K$ algebraisch über $k$. Wir setzen
\[
 F := k(x_1, \ldots, x_r).
\]

Für alle $r < k \leq n$ ist die Familie $x_1, \ldots, x_r, x_k$ algebraisch abgängig über $k$ ist, also $x_k$ algebraisch über $F$. Daher ist die Körpererweiterung
\[
 F \subseteq K = F(x_{r+1}, \ldots, x_n)
\]
algebraisch. Da sie von endlich vielen algebraischen Elementen erzeugt wird, ist sie sogar endlich, d.h. $K$ ist als $F$-Vektorraum endlich. Inbesondere ergibt sich daher aus Aufgabe 3, dass $F$ als $k$-Algebra endlich erzeugt ist. Es gibt also $y_1, \ldots, y_s \in F$ mit
\[
 F = k[y_1, \ldots, y_s].
\]
Da die $x_1, \ldots, x_r$ algebraisch unabhängig sind, ist $k[x_1, \ldots, x_r]$ isomorph zum Polynomring in $r$ Variablen über $k$, und $F$ zu dessen Quotientenkörper. Daher gibt es für alle $i = 0, \ldots, s$ Polynome $f_i, g_i \in k[x_1, \ldots, x_r]$ mit $y_i = f_i/g_i$.

Es sei $\bar{k}$ ein algebraischer Abschluss von $k$. Wir setzen
\[
 X := \mc{V}(\{g_i \mid 1 \leq i \leq s\}) = \mc{V}\left(\prod_{i=1}^s g_i\right).
\]
Wir bemerken, dass $X \neq \bar{k}^r$, denn die Inklusion
\[
 k[x_1, \ldots, x_r] \hookrightarrow \bar{k}[x_1, \ldots, x_r] \cong \mc{P}(\bar{k}).
\]
ist ein injektiver, unitärer Ringhomomorphismus, und da $g_i \neq 0$ für alle $1 \leq i \leq s$ ist auch $\prod_{i=1}^s g_i \neq 0$. Also gibt es $z \in \bar{k}^r$ mit $\prod_{i=1}^n g_i (z) \neq 0$.

Wir betrachten die Abbildung
\[
 \varphi : F \to \Abb\left(\bar{k}^r \smallsetminus X, \bar{k}\right),
 \left[ \frac{f}{g} \right] \mapsto \left( x \mapsto \frac{f(x)}{g(x)} \right),
\]
wobei die Wohldefiniertheit klar ist. Für alle $1 \leq i \leq r$ sei
\[
 h_i \in k[x_i] \subseteq k[x_1, \ldots, x_r] \subseteq F
\]
das Minimalpolynom von $z_i$ über $k$ (da $x_i$ transzendend über $k$ ist, können wir $k[x_i]$ als Polynomring über $k$ verstehen). Wir setzten $h := \varphi(\prod_{i=1}^r h_i)$. Da $h \neq 0$ ist $h$ in $F$ invertierbar, also auch $\varphi(h)$ in $\Abb(\bar{k}^r \smallsetminus X, \bar{k})$. Dies steht jedoch im Widerspruch dazu, dass $\varphi(h)(z) = 0$.





\end{document}
