\documentclass[a4paper,10pt]{article}
%\documentclass[a4paper,10pt]{scrartcl}

\usepackage{xltxtra}
\usepackage{../mystyle}

\setromanfont[Mapping=tex-text]{Linux Libertine O}
% \setsansfont[Mapping=tex-text]{DejaVu Sans}
% \setmonofont[Mapping=tex-text]{DejaVu Sans Mono}

\title{\sc Algebra I \\ \Large Blatt 8}
\author{Thorben Kastenholz \\ Jendrik Stelzner}
\date{\today}

\begin{document}
\maketitle

\section{}

\begin{lem}\label{lem: Äquivalente Definitionen von noethersch}
 Es sei $R$ ein Ring und $M$ ein $R$-Modul. Dann sind äquivalent:
 \begin{enumerate}[i)]
  \item $M$ ist noethersch, d.h. jede aufsteigende Kette von Untermoduln von $M$
  \[
   M_0 \subseteq M_1 \subseteq M_2 \subseteq \ldots
  \]
  stabilisiert.
  \item Jeder Untermodul von $M$ ist endlich erzeugt über $R$.
 \end{enumerate}
 Insbesondere ist ein kommutiver Ring $R$ genau dann noethersch, wenn jede aufsteigende Kette von Idealen
 \[
  I_0 \subseteq I_1 \subseteq I_2 \subseteq I_3 \subseteq \ldots
 \]
 in $R$ stabilisiert.
\end{lem}
\begin{proof}
 Angenommen, $M$ ist noethersch. Es sei $M' \subseteq M$ ein Untermodul. Dann definieren wir eine eine aufsteigende Folge von Untermoduln von $M'$ wie folgt: Wir beginnen mit $M_0 := 0$. Ist $M_i$ definiert und $M_i \neq M'$, so gibt es $m_{i+1} \in M' \smallsetminus M_i$, und wir setzen $M_{i+1} := M_i + R m_{i+1}$; ansonsten setzen wir $M_{i+1} := M_i = M'$. Da $M$ noethersch ist, stabilisiert die aufsteigende Kette
 \[
  0 = M_0 \subsetneq M_1 \subsetneq M_2 \subsetneq M_3 \subsetneq \ldots
 \]
 von Untermoduln von $M$. Nach Konstruktion der $M_i$ gibt es daher $n \in \N$ mit
 \[
  M' = M_n = Rm_1 + \ldots + Rm_n = (m_1, \ldots, m_n).
 \]
 Das zeigt, dass $M'$ ein endlich erzeugter $R$-Modul ist.
 
 Sei andererseits jeder Untermodul von $M$ endlich erzeugt über $R$. Für eine aufsteigende Kette
 \[
  M_0 \subseteq M_1 \subseteq M_2 \subseteq \ldots
 \]
 von Untermoduln von $M$ setzen wir
 \[
  M' := \bigcup_{k \in \N} M_k.
 \]
 $M'$ ist ein Untermodul von $M$ und somit endlich erzeugt. Nach Annahme gibt es daher $m_1, \ldots, m_n \in M'$ mit
 \[
  M' = (m_1, \ldots, m_n).
 \]
 Nach Definition von $M'$ gibt es ein $N \in \N$ mit $m_1, \ldots, m_n \in M_N$. Es ist daher $M_N = M$ und somit auch $M_k = M$ für alle $k \geq N$. Also stabilisiert die Kette.
\end{proof}




\subsection{}
Da $k$ kommutativ ist und nur zwei Ideale enthält, ist $k$ offenbar noethersch. Induktiv ergibt sich damit aus dem Hilbertschen Basissatz direkt, dass auch $k[x_1, \ldots, x_n]$ für alle $n \in \N$ noethersch ist.


\subsection{}
Es sei $R$ ein kommutativer noetherscher Ring. Angenommen, $R[X]$ ist nicht noethersch. Nach Lemma \ref{lem: Äquivalente Definitionen von noethersch} gibt es dann ein Ideal $I \subseteq R[X]$ das nicht endlich erzeugt über $R[X]$ ist. (Inbesondere ist $I \neq 0$.)

Wir definieren eine Folge $(f_i)_{i \geq 1}$ von Polynomen $f_i \in I$ wie folgt: Wir wählen $f_1 \in I \smallsetminus \{0\}$ mit minimalen Grad. Ist $f_i$ definiert, so ist, da $I$ nicht endlich erzeugt ist,
\[
 (f_1, \ldots, f_i) \neq I.
\]
Es sei dann $f_{i+1} \in I \smallsetminus (f_1, \ldots, f_i)$ vom minimalen Grad. Man bemerke, dass stets $\deg f_i \leq \deg f_{i+1}$.

Für alle $i \geq 1$ definieren wir $a_i \in R$ als den Leitkoeffizienten von $f_i$ und setzen
\[
 J_i := (a_1, \ldots, a_i) \subseteq R.
\]
Da $R$ noethersch ist, stabilisiert die aufsteigende Kette von Idealen
\[
 J_1 \subseteq J_2 \subseteq J_3 \subseteq \ldots
\]
Es gibt also ein $n \geq 1$ mit
\[
 (a_1, \ldots, a_{n+1}) = J_{n+1} = J_n = (a_1, \ldots, a_n),
\]
was äquivalent dazu ist, dass $a_{n+1} \in (a_1, \ldots, a_n)$. Es gibt also $r_1, \ldots, r_n \in R$ mit
\[
 a_{n+1} = \sum_{i=1}^n r_i a_i.
\]
Deshalb ist
\[
 g := \sum_{i=1}^n r_i f_i \cdot X^{\deg f_{n+1} - \deg f_i} \in (f_1, \ldots, f_n)
\]
ein Polynom mit gleichem Grad und gleichen Leitkoeffizienten wie $f_{n+1}$. Da $f_{n+1} \not\in (f_0, \ldots, f_n)$ ist auch $f_{n+1}-g \not\in (f_1, \ldots, f_n)$. Da $\deg (f_{n+1}-g) < \deg f_{n+1}$ steht dies im Widerspruch zur Gradminimimalität von $f_{n+1}$.

Es ist also jedes Ideal in $R[X]$ endlich erzeugt, und somit $R[X]$ somit noethersch.






\section{}


\subsection{}
Es sei
\begin{center}
 \begin{tikzpicture}[auto, node distance = 4em]
  \node (01) {$0$};
  \node (M') [right of = 01] {$M'$};
  \node (M) [right of = M'] {$M$};
  \node (M'') [right of = M] {$M''$};
  \node (02) [right of = M''] {$0$};
  \draw[->] (01) to (M');
  \draw[->] (M') to node {$i$} (M);
  \draw[->] (M) to node {$\pi$} (M'');
  \draw[->] (M'') to (02);
 \end{tikzpicture}
\end{center}
eine kurze exakte Sequenz von $R$-Moduln. Wir können o.B.d.A. davon ausgehen, dass $M' \subseteq M$ ein Untermodul ist, $i$ die kanonische Inklusion, $M'' = M/M'$ und $\pi$ die kanonische Projektion.

Angenommen, $M$ ist noethersch. Jede aufsteigende Ketten von Untermoduln
\[
 M_0 \subseteq M_1 \subseteq M_2 \subseteq \ldots
\]
von $M'$ ist auch eine aufsteigende Kette von Untermoduln von $M$ und stabilisiert somit. Also ist $M'$ noethersch. Jede aufsteigende Kette von Untermoduln
\[
 N_0 \subseteq N_1 \subseteq N_2 \subseteq \ldots
\]
von $M'' = M/M'$ liefert eine aufsteigende Kette von Untermoduln
\[
 \pi^{-1}(N_0) \subseteq \pi^{-1}(N_1) \subseteq \pi^{-1}(N_2) \subseteq \ldots
\]
von $M$, die $M'$ enthalten. Da $M$ noethersch ist stabilisiert diese Kette, und da $\pi$ surjektiv ist, und somit
\[
 \pi(\pi^{-1}(N_i)) = N_i \text{ für alle } i \in \N
\]
stabilisiert auch die Kette in $M''$. Also ist $M''$ noethersch.

Angenommen, $M'$ und $M''$ sind noethersch. Es sei dann
\[
 M_0 \subseteq M_1 \subseteq M_2 \subseteq \ldots
\]
eine aufsteigende Kette von Untermoduln von $M$. Für alle $i \in \N$ setzen wir
\[
 M'_i := M' \cap M_i \text{ und } M''_i := \pi(M_i)
\]
und erhalten so aufsteigende Ketten von Untermoduln von $M'$ und $M''$. Da diese noethersch sind stabilisieren dies Ketten. Es gibt also $N \in \N$ mit
\[
 M'_n = M'_N \text{ und } M''_n = M''_N \text{ für alle } n \geq N.
\]
Daher ist auch $M_n = M_N$ für alle $n \geq N$. (Denn für alle $n \geq N$ enthalten wir das folgende kommutive Diagramm mit exakten Zeilen.
\begin{center}
 \begin{tikzpicture}[auto, node distance = 5em]
 \node (01) {$0$};
 \node (M'_N) [right of = 01] {$M'_N$};
 \node (M_N) [right of = M'_N] {$M_N$};
 \node (M''_N) [right of = M_N] {$M''_N$};
 \node (02) [right of = M''_N] {$0$};
 \node (03) [below of = 01] {$0$};
 \node (M'_n) [right of = 03] {$M'_n$};
 \node (M_n) [right of = M'_n] {$M_n$};
 \node (M''_n) [right of = M_n] {$M''_n$};
 \node (04) [right of = M''_n] {$0$};
 \draw[->] (01) to (M'_N);
 \draw[->] (M'_N) to node {$i$} (M_N);
 \draw[->] (M_N) to node {$\pi$} (M''_N);
 \draw[->] (M''_N) to (02);
 \draw[double equal sign distance] (01) to (03);
 \draw[double equal sign distance] (M'_N) to (M'_n);
 \draw[left hook->] (M_N) to (M_n);
 \draw[double equal sign distance] (M''_N) to (M''_n);
 \draw[double equal sign distance] (02) to (04);
 \draw[->] (03) to (M'_n);
 \draw[->] (M'_n) to node {$i$} (M_n);
 \draw[->] (M_n) to node {$\pi$} (M''_n);
 \draw[->] (M''_n) to (04);
 \end{tikzpicture}
\end{center}
Nach dem Fünferlemma ist die Inklusion bereits eine Gleichheit.)


























\end{document}
